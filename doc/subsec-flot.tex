\subsection{Flots}
Nous avons désormais toutes les structures nécéssaires à la construction de notre graphe de flot. Pour le moment, il s'agira de 
le construire naïvement, l'on reviendra plus tard sur les optimisations possibles. Ainsi, une manière simple de représenter ce 
graphe est de considérer l'ensemble des couples des étiquettes de blocs qui indiqueront un arc du premier élément au second. 
Pour ce faire, considérons l'application suivante, 

\begin{align*}
	\varrho : Stm &\longrightarrow \mathcal{P}(\mathbb{N}^2) \\
	\sassign{Id}{Exp_a} &\longmapsto \emptyset \\
	\sskip &\longmapsto \emptyset \\
	\sseq{s_1}{s_2} &\longmapsto \varrho(s_1) \cup \varrho(s_2) \cup [\phi(s_1)\times\{\iota(s_2)\}] \\
	\ifthenelse{b^l}{s_1}{s_2} &\longmapsto \varrho(s_1)\cup\varrho(s_2)\cup(l, \iota(s_1))\cup(l, \iota(s_2)) \\
	\whiledo{b^l}{s} &\longmapsto \varrho(s)\cup(l,\iota(s))\cup[\phi(s)\times\{l\}]
\end{align*}
On introduira aussi un accès à l'ensemble des successeurs d'un bloc par la fonction,
\begin{align*}
	succ : \mathbb{N} &\longrightarrow \mathcal{P}(\mathbb{N})\\
	l &\longmapsto \{l' \in \mathbb{N} \mid (l, l') \in \mathcal{G}_V\},
\end{align*}
où $\mathcal{G}_V$ est l'ensemble des arcs du graphe de flot de contrôle du programme.