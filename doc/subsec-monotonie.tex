\subsection{Monotonie}
La monotonie des ensembles de flot de données nous permet de déterminer l'existence d'un point fixe sur notre
fonction. Cela sera donc utile pour fournir un algorithme qui termine, lors de la construction de ces ensembles, 
étant donné que l'ensemble des variables d'un programme est supposé fini.
\\
\\
On se consacre, dans cette partie, à montrer la monotonie de notre ensemble $LIVE_{in}[l]$ pour toute étiquette de blocs. Posons la fonction,
\begin{align*}
	f_l : \mathcal{P}(V) &\longrightarrow \mathcal{P}(V) \\
	\mathcal{O} &\longmapsto GEN[l] \cup (\mathcal{O} - KILL[l])
\end{align*}
Soient $k, k' \in \mathcal{P}(V)$ tels que $k \subseteq k'$, alors on a,
\begin{align*}
	k - KILL[l] &\subseteq k' - KILL[l] \text{ et,}\\
	GEN[l] \cup (k - KILL[l]) &\subseteq GEN[l] \cup (k' - KILL[l]).
\end{align*}
Donc $f_l(k) \subseteq f_l(k')$ donc $f_l$ est monotone. On a ainsi que $LIVE_{in}[l]$, dépendant de $LIVE_{out}[l]$, est monotone.