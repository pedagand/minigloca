\subsection{Monotonie}
La monotonie des ensembles de flot de données nous permet de déterminer l'existence d'un point fixe sur notre
fonction. Cela sera donc utile pour fournir un algorithme qui termine, lors de la construction de ces ensembles, 
étant donné que l'ensemble des variables d'un programme est supposé fini.
\\
\\
On se consacre, dans cette partie, à montrer la monotonie de nos ensembles $LIVE_{in}[l]$ et $LIVE_{out}[l]$
pour toute étiquette $l$ de blocs. Posons la fonction,
\begin{align*}
	f_l : \mathcal{P}(V) &\longrightarrow \mathcal{P}(V) \\
	\mathcal{O} &\longmapsto GEN[l] \cup (\mathcal{O} - KILL[l])
\end{align*}
Soient $k, k' \in \mathcal{P}(V)$ tels que $k \subseteq k'$, alors on a,
\begin{align*}
	k - KILL[l] &\subseteq k' - KILL[l] \text{ et,}\\
	GEN[l] \cup (k - KILL[l]) &\subseteq GEN[l] \cup (k' - KILL[l]).
\end{align*}
Donc $f_l(k) \subseteq f_l(k')$ ce qui implique $f_l$ monotone. Comme $LIVE_{out}[l]$ est une union de tous les $LIVE_{in}$
de ses successeurs et que $f_l$ est une fonction monotone, on a que $LIVE_{out}[l]$ est monotone pour tout $l$, par récurrence
sur la monotonie de l'union. Il vient de plus que $LIVE_{in}[l]$, dépendant de $LIVE_{out}[l]$, est monotone.
Ainsi ces deux fonctions dépendants de l'étiquette d'un bloc sont monotones et produisent des ensembles finis, étant donné
la finitude de $V$ dans notre cas.