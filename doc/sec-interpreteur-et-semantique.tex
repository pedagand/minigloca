\section{Interpréteur et sémantique}
Dans ce chapitre, on s'atèle à décrire l'interpréteur ainsi que la sémantique sur notre petit langage impératif. 
Dans ce cadre là, on définit l'état du programme par une bijection entre l'identifiant des variables et leur valeur, 
ici un entier:
\[\sigma : \mathbb{V} \longrightarrow \mathbb{Z}\]
On introduire aussi la bijection $\mu_B$,
\[\mu_B : \mathcal{B} \longrightarrow \mathbb{B}\]
où $\mathcal{B} = \{\textbf{true}$, \textbf{false}\} les valeurs booléennes de notre langage et $\mathbb{B} = \{\top, \bot\}$ 
l'ensemble des valeurs booléennes natives à OCaml. Pour la suite, on considèrera l'ensemble des états $\mathcal{S} = \mathbb{Z}^\mathbb{V}$.
Les états décriront l'évolution de l'exécution du programme.

\subsection{Interpréteur}
Sur les expressions arithmétiques, définies dans le chapitre 1, on se donne la fonction suivante:
\begin{align*}
	[\![Exp_a]\!]^A : \mathcal{S} &\longrightarrow \mathbb{Z}\\
	[\![Id]\!]^A \sigma &\longmapsto \sigma(Id)\\
	[\![a_1+a_2]\!]^A \sigma &\longmapsto [\![a_1]\!]^A(\sigma)\text{ }\hat{+}\text{ }[\![a_2]\!]^A(\sigma)\\
\end{align*}
Cette dernière est aussi définie respectivement aux autres opérateurs arithmétiques, introduits dans le chapitre précédent.
De la même manière, on définit la fonction qui suit sur les expressions booléennes:
\begin{align*}
	[\![Exp_b]\!]^{B} : \mathcal{S} &\longrightarrow \mathbb{B}\\	
	[\![b]\!]^{B} \sigma &\longmapsto \mu(b)\\
	[\![a_1 = a_2]\!]^B \sigma &\longmapsto [\![a_1]\!]^{A}(\sigma)\text{ }\hat{=}\text{ }[\![a_2]\!]^A(\sigma)\\
	[\![b_1 \wedge b_2]\!]^B \sigma &\longmapsto [\![b_1]\!]^B(\sigma)\text{ }\hat{\wedge}\text{ }[\![b_2]\!]^B(\sigma)\\
	[\![\neg b]\!]^B \sigma &\longmapsto \hat{\neg}[\![b]\!]^B(\sigma)
\end{align*}
Les opérateurs de la forme $\hat{op}$ représentent les opérateurs natifs à OCaml.
\subsection{Sémantique}
Maintenant que nous avons correctement défini l'interpréteur, il est possible de construire la sémantique du langage.
La logique de Hoare est un modèle formel qui nous permet la correction rigoureuse de nos programmes. Soient $P, Q$ des 
prédicats et $C$ une déclaration. Alors le triplet de Hoare est défini comme tel,

\[\{P\}C\{Q\}\]
Il est ensuite possible de développer un ensemble de règles logiques. Pour des raisons de commodités, on adaptera légèrement 
la syntaxe de la déclaration des règles, en prenant en compte nos état et déclaration, cela prendra donc la forme suivante,

\[Stm, \mathcal{S} \longrightarrow Stm', \mathcal{S}'\]
\\
où $Stm$ est la première déclaration, $Stm'$ celle qui suit après son exécution, $\mathcal{S}$ l'état initial et $\mathcal{S}'$ 
l'état successeur. Commençons par la déclaration vide,

\begin{mathpar}
	\inferrule*[]
	{ }
	{\textit{Skip}, \sigma_A \longrightarrow \emptyset, \sigma_A}
\end{mathpar}
Poursuivons avec la déclaration de l'assignation,
\begin{mathpar}
	\inferrule*[]
	{ }
	{Id:=a,\sigma_A \longrightarrow \emptyset, \sigma_A' : Id \longmapsto [\![a]\!]^A(\sigma_A)}
\end{mathpar}
Ici, le nouvel état $\sigma_A'$ lie l'identifiant de la variable assignée, à la valeur de l'expression arithmétique $[\![a]\!]^A$.
Poursuivons avec la règle de la séquence entre deux déclarations, d'une part si la première termine,
\begin{mathpar}
	\inferrule*[]
	{Stm_1,\sigma \longrightarrow \emptyset,\sigma'}
	{Stm_1;Stm_2,\sigma \longrightarrow Stm_2, \sigma'}
\end{mathpar}
D'autre part, si la première ne termine pas,
\begin{mathpar}
	\inferrule*[]
	{Stm_1,\sigma \longrightarrow Stm_1',\sigma'}
	{Stm_1;Stm_2,\sigma \longrightarrow Stm_1';Stm_2,\sigma'}
\end{mathpar}
La condition peut se formaliser de la sorte dans le cas où la garde est vérifiée,
\begin{mathpar}
	\inferrule*[]
	{[\![b]\!]^B(\sigma_B)}
	{\text{if }b\text{ then }Stm_1\text{ else }Stm_2, \sigma_B \longrightarrow Stm_1, \sigma_B}
\end{mathpar}
Dans le cas où elle ne l'est pas,
\begin{mathpar}
	\inferrule*[]
	{\neg[\![b]\!]^B(\sigma_B)}
	{\text{if }b\text{ then }Stm_1\text{ else }Stm_2, \sigma_B \longrightarrow Stm_2, \sigma_B}
\end{mathpar}
La dernière de nos déclarations est la boucle while, celle-ci peut en fait être décrite grâce à la déclaration de la condition de la manière suivante,
\[
\text{while } b \text{ do }Stm
\equiv_d
\text{if }b\text{ then }Stm\text{ while }b\text{ do }Stm\text{ else }\textit{Skip}
\]
Ainsi, on peut déclarer la règle qui suit pour la déclaration while,
\begin{mathpar}
	\inferrule*[]
	{[\![b]\!]^B(\sigma_B)}
	{\text{while }b\text{ do }Stm, \sigma_B \longrightarrow Stm\text{ ; while }b\text{ do }Stm,\sigma_B}
\end{mathpar}