\documentclass[a4paper, 12pt]{article}

\usepackage[francais]{babel}
\usepackage[T1]{fontenc}
\usepackage{amssymb}
\usepackage{amsmath}
\usepackage{mathpartir}
\usepackage{tikz}
\usepackage{listings}
\usepackage{algorithm}
\usepackage{algpseudocode}
\usepackage{color}
\usepackage{mathtools} % for 'dcases*' env.

\definecolor{dkgreen}{rgb}{0,0.6,0}
\definecolor{gray}{rgb}{0.5,0.5,0.5}
\definecolor{mauve}{rgb}{0.58,0,0.82}

\lstset{frame=tb,
	language=caml,
	columns=[c]fixed,
	basicstyle=\small\ttfamily,
	keywordstyle=\bfseries,
	upquote=true,
	commentstyle=,
	breaklines=true,
	showstringspaces=false,
	stringstyle=\color{green}
}

\title{Minigloca}
\author{Vladislas de Haldat}

\begin{document}

\newcommand{\deftype}[2]{
    ${#1}\rightarrow$\foreach[count=\i] \e in {#2} {
            \ifnum\i>1$\mid$\fi
            \space${\e}$\\
        }
}

\newcommand{\definition}[1] {
    \textbf{\textit{Définition --\space}}{#1}
}

\newcommand{\theorem}[1] {
    \textbf{\textit{Théorème --\space}}{#1}
}

\newcommand{\sassign}[2] {
    {#1}\space:=\space{#2}
}

\newcommand{\sseq}[2] {
    {#1}\space;\space{#2}
}

\newcommand{\sskip}{
    \textit{Skip}
}

\newcommand{\sifthenelse}[3] {
    \text{if }{#1}\text{ then }{#2}\text{ else }{#3}
}

\newcommand{\swhiledo}[2] {
    \text{while }{#1}\text{ do }{#2}
}

\newcommand{\intrfun}[2] {
[\![{#1}]\!]^{#2}
}

\newcommand{\intr}[3] {
    \intrfun{#1}{#2} ({#3})
}

\newcommand{\semanticd}[4] {
    {#1},\space{#2}\longrightarrow{#3},\space{#4}
}

\newcommand{\srule}[2] {
    \begin{mathpar}
        \inferrule*[]
        {#1}
        {#2}
    \end{mathpar}
}

\newenvironment{dtype}[1]{
    \newcommand{\inlinekind}[1]{##1}
    \newcommand{\kind}[1]{\mid##1}
    \newcommand{\akind}[1]{&\kind{##1}}
    \newcommand{\with}[1]{&##1\\}
    \csname align*\endcsname
    #1 &\rightarrow}{
    \csname endalign*\endcsname
}

\maketitle
\newpage
\tableofcontents
\newpage

\section{Un langage impératif simple}
Pour commencer, définissons une syntaxe abstraite minimale que nous utiliserons tout le long de cette étude.
Cette syntaxe se composera de trois blocs fondamentaux que sont les expressions arithmétiques, les expressions
booléennes ainsi que les déclarations. Nous n'incluons pas pour le moment la déclaration de routines au sein de cette syntaxe.

\subsection{Expressions arithmétiques}
Les expressions arithmétiques sont définies sur l'ensemble des entiers relatifs. On se donne les opérateurs
de l'addition, de la soustraction ainsi que de la multiplication. À ces opérateurs l'on pourra appliquer des
entiers ainsi que des identifiants de variables.

% \begin{dtype}{Int}
%   \inlinekind{n}\with{n\in\mathbb{Z}}
% \end{dtype}
% \begin{dtype}{Id}
%   \inlinekind{x}
%   \kind{y}
%   \kind{z}
%   \kind{\ldots}
% \end{dtype}
% \begin{dtype}{Exp_a}
%   \inlinekind{Int}
%   \kind{Id}
%   \kind{op_A (a_1, a_2)}
%   \with{op_A \in\{+, -, \times\}}
% \end{dtype}

\begin{align*}
  Int   & \rightarrow n                               & n \in \mathbb{Z}          \\
  Id    & \rightarrow x \mid y \mid z \mid \cdots                                 \\
  Exp_a & \rightarrow Int \mid Id \mid op_A(a_1, a_2) & op_A \in \{+, -, \times\}
\end{align*}
L'identifiant d'une variable est, de manière générale, une chaîne de caractères.

\subsection{Expressions booléennes}
Les expressions booléennes nous permettent d'introduire la comparaison entre deux expressions arithmétiques,
ainsi que les opérateurs booléens sur les expressions booléennes.

\begin{dtype}{Exp_b}
  \inlinekind{\textbf{true}}
  \kind{\textbf{false}}\\
  \akind{op_R (a_1, a_2)}\with{op_R\in\{<, =\}}
  \akind{op_B (b_1, b_2)}\with{op_B\in\{\wedge, \vee\}}
  \akind{\neg{b}}
\end{dtype}

\subsection{Déclarations}
Les déclarations sont définies de la manière suivante:

\begin{dtype}{Stm}
  \inlinekind{\sassign{Id}{Exp_a}}\\
  \akind{\sseq{s_1}{s_2}}\\
  \akind{ }\\
  \akind{\sifthenelse{b}{s_1}{s_2}}\\
  \akind{\swhiledo{b}{s}}
\end{dtype}

\subsection{Exemple}
Voici un premier exemple sur ce langage
\begin{lstlisting}
a := 1;
b := 20;

if a = 3 then
	c := 4
else
	c := 6
endif;

while b < 100 do
	b := b + 1
done
\end{lstlisting}

\subsection{Arbre de syntaxe abstraite}
Ce modèle de représentation permet de construire un arbre de syntaxe abstraite (AST) du programme en question,
cette structure étant un moyen pratique de le représenter. Cela nous servira particulièrement lors des différentes
analyses qui seront effectuées, et notamment l'analyse par flot de contrôle. Cette dernière ne requièrant pas
la connaissance de l'ordre d'exécution du programme, les AST sont tout-à-fait adaptés!

\begin{center}\input{ast.fig}\end{center}

\subsection{Graphe de flot de contrôle}
Un graphe de flot de contrôle est un graphe orienté dont les noeuds représentent une déclaration et les arcs un
flot de contrôle. Chaque noeud est une unique opération de notre programme. Dans notre cas, le graphe de flot de
contrôle aura un seul point d'entrée et un seul point de sortie. Ils sont considérés comme des noeuds ``sans opération''.
De manière à construire ce graphe, on peut assigner à chaques déclarations procurant une unique exécution, un label.
Dans notre cas, ces déclarations sont l'assignation, la condition ainsi que la boucle. En voici quelques exemples:

\begin{center}\input{cfg.fig}\end{center}

\subsection{Implémentation}
Détails sur l'implémentation.

\section{Interpréteur et sémantique}
Dans ce chapitre, on s'atèle à décrire l'interpréteur ainsi que la sémantique sur notre petit langage impératif. 
Dans ce cadre là, on définit l'état du programme par une bijection entre l'identifiant des variables et leur valeur, 
ici un entier:
\[\sigma : \mathbb{V} \longrightarrow \mathbb{Z}\]
On introduire aussi la bijection $\mu_B$,
\[\mu_B : \mathcal{B} \longrightarrow \mathbb{B}\]
où $\mathcal{B} = \{\textbf{true}$, \textbf{false}\} les valeurs booléennes de notre langage et $\mathbb{B} = \{\top, \bot\}$ 
l'ensemble des valeurs booléennes natives à OCaml. Pour la suite, on considèrera l'ensemble des états $\mathcal{S} = \mathbb{Z}^\mathbb{V}$.
Les états décriront l'évolution de l'exécution du programme.

\subsection{Interpréteur}
Sur les expressions arithmétiques, définies dans le chapitre 1, on se donne la fonction suivante:
\begin{align*}
	\intrfun{Exp_a}{A}:\mathcal{S}&\longrightarrow\mathbb{Z}\\
	\intrfun{Id}{A}\sigma&\longmapsto\sigma(Id)\\
	\intrfun{a_1+a_2}{A}\sigma&\longmapsto\intr{a_1}{A}{\sigma}\text{ }\hat{+}\text{ }\intr{a_2}{A}{\sigma}\\
\end{align*}
Cette dernière est aussi définie respectivement aux autres opérateurs arithmétiques, introduits dans le chapitre précédent.
De la même manière, on définit la fonction qui suit sur les expressions booléennes:
\begin{align*}
	\intrfun{Exp_b}{B}:\mathcal{S}&\longrightarrow\mathbb{B}\\	
	\intrfun{b}{B}\sigma&\longmapsto\mu(b)\\
	\intrfun{a_1=a_2}{B}\sigma &\longmapsto \intr{a_1}{A}{\sigma}\text{ }\hat{=}\text{ }\intr{a_2}{A}{\sigma}\\
	\intrfun{b_1\wedge b_2}{B}\sigma&\longmapsto \intr{b_1}{B}{\sigma}\text{ }\hat{\wedge}\text{ }\intr{b_2}{B}{\sigma}\\
	\intrfun{\neg b}{B}\sigma&\longmapsto\hat{\neg}\intr{b}{B}{\sigma}
\end{align*}
Les opérateurs de la forme $\hat{op}$ représentent les opérateurs natifs à OCaml.

\subsection{Sémantique}
Maintenant que nous avons correctement défini l'interpréteur, il est possible 
de construire la sémantique du langage. Il est ensuite possible de développer 
un ensemble de règles logiques. La syntaxe de la déclaration des règles,
prenant en compte nos état et déclaration, sera donc de la forme suivante,

\[Stm, \mathcal{S} \longrightarrow Stm', \mathcal{S}'\]
\\
où $Stm$ est la première déclaration, $Stm'$ celle qui suit après son exécution, 
$\mathcal{S}$ l'état initial et $\mathcal{S}'$  l'état successeur. Commençons 
par la déclaration vide,

\srule{ }{\semanticd{\sskip}{\sigma}{\emptyset}{\sigma}}

Poursuivons avec la déclaration de l'assignation,
\srule{ }{\semanticd{\sassign{Id}{a}}{\sigma}{\emptyset}{\sigma'[Id\longmapsto\intr{a}{A}{\sigma}]}}
Ici, le nouvel état $\sigma'$ lie l'identifiant de la variable assignée, à la 
valeur de l'expression arithmétique $\intrfun{a}{A}$. Poursuivons avec la règle 
de la séquence entre deux déclarations, d'une part si la première termine,
\srule{\semanticd{Stm_1}{\sigma}{\emptyset}{\sigma'}}
{\semanticd{\sseq{Stm_1}{Stm_2}}{\sigma}{Stm_2}{\sigma'}}
D'autre part, si la première ne termine pas,
\srule{\semanticd{Stm_1}{\sigma}{Stm_1'}{\sigma'}}
{\semanticd{\sseq{Stm_1}{Stm_2}}{\sigma}{\sseq{Stm_1'}{Stm_2}}{\sigma'}}
La condition peut se formaliser de la sorte dans le cas où la garde est vérifiée,
\srule{\intr{b}{B}{\sigma}}
{\semanticd{\sifthenelse{b}{Stm_1}{Stm_2}}{\sigma}{Stm_1}{\sigma}}
Dans le cas où elle ne l'est pas,
\srule{\neg\intr{b}{B}{\sigma}}
{\semanticd{\sifthenelse{b}{Stm_1}{Stm_2}}{\sigma}{Stm_2}{\sigma}}
La dernière de nos déclarations est la boucle while, celle-ci peut en fait être 
décrite grâce à la déclaration de la condition de la manière suivante,
\[
\swhiledo{b}{Stm}
\equiv_d
\sifthenelse{b}{\swhiledo{b}{Stm}}{\sskip}
\]
Ainsi, on peut déclarer la règle qui suit pour la déclaration while,
\srule{\intr{b}{B}{\sigma}}
{\semanticd{\swhiledo{b}{Stm}}{\sigma}{\sseq{Stm}{\swhiledo{b}{Stm}}}{\sigma}}

\section{Prérequis à l'analyse}
De manière à pouvoir travailler avec les analyses de flot de données et de flot de contrôle, 
il nous est au préalable nécessaire d'approfondir le principe d'étiquettage déjà abordé à la fin du premier chapitre.

\subsection{Étiquettage}
Avant tout, introduisons la fonction suivante,

\[
	\lambda: Stm \longrightarrow \mathbb{N}
\]
\newline
qui prend une déclaration $s$ et retourne une étiquette unique.
\newline
\newline
\definition{Soient $s \in Stm$ et $(l_n)_{n\in\mathbb{N}}$ une liste d'étiquettes, $s$ est dite bien formée si et seulement}
si $\forall i \in \mathbb{N}, \forall j \in \mathbb{N}$ tel que $i \ne j$ alors $l_i \ne l_j$.
\newline
\newline
Étant donné un programme $P$ défini par l'ensemble de ses déclarations atomiques, on a alors que 
$\pi \in \mathcal{P}(P)$, un sous bloc de ce programme, n'admet qu'une seule entrée et une ou plusieurs sorties. 
On peut maintenant définir la fonction $\iota$, qui retournera la première étiquette rencontrée dans une déclaration,

\begin{align*}
	\iota : Stm &\longrightarrow \mathbb{N}\\
	(\sassign{Id}{Exp_a})^l &\longmapsto l\\
	\sseq{s_1}{s_2} &\longmapsto \iota(s_1)\\
	{\sskip}^l &\longmapsto l\\
	\sifthenelse{b^l}{s_1}{s_2} &\longmapsto l\\
	\swhiledo{b^l}{s} &\longmapsto l
\end{align*}
Comme expliqué précédemment, il est aussi nécessaire de déclarer une fonction $\phi$ qui retournera l'ensemble des étiquettes finales 
à la fin d'un bloc $\pi$.

\begin{align*}
	\phi : Stm &\longrightarrow \mathcal{P}(\mathbb{N})\\
	(\sassign{Id}{Exp_a})^l &\longmapsto \{l\}\\
	\sseq{s_1}{s_2} &\longmapsto \phi(s_2)\\
	\sskip^l &\longmapsto \{l\}\\
	\sifthenelse{b^l}{s_1}{s_2} &\longmapsto \phi(s_1) \cup \phi(s_2)\\
	\swhiledo{b^l}{s} &\longmapsto \{l\}
\end{align*}
\subsection{Blocs}
De manière à faciliter l'analyse d'un programme, il est utile de partitionner et de factoriser notre représentation du code. 
Les blocs ont été évoqués dans la partie précédente comme des sous-parties de l'ensemble des déclarations atomiques d'un programme. 
On peut désormais les définir de la sorte,

\begin{dtype}{Block}
	\inlinekind{\sassign{Id}{Exp_a}}\\
	\akind{Exp_b}\\
	\akind{\sskip}
\end{dtype}
Maintenant les blocs correctement définis, il nous faut pouvoir les lier à une étiquette. Pour ce faire, on se donne $\beta$ définie par,

\begin{align*}
	\beta : Stm &\longrightarrow \mathcal{P}(\mathbb{N} \times Block)\\
	(\sassign{Id}{Exp_a})^l&\longmapsto\{(l, Id := Exp_a)\}\\
	\sseq{s_1}{s_2} &\longmapsto \beta(s_1) \cup \beta(s_2)\\
	\sskip^l &\longmapsto \{(l, \textit{Skip})\}\\
	\sifthenelse{b^l}{s_1}{s_2} &\longmapsto \{(l, b)\}\cup \beta(s_1)\cup\beta(s_2)\\
	\swhiledo{b^l}{s} &\longmapsto \{(l, b)\}\cup\beta(s)
\end{align*}
À partir de là, il est possible de formaliser les graphes orientés de flot. Les blocs en représentent les noeuds, et le passage 
vers le bloc suivant est représenté par un arc.

\subsection{Flots}
Nous avons désormais toutes les structures nécéssaires à la construction de notre graphe de flot. Pour le moment, il s'agira de 
le construire naïvement, l'on reviendra plus tard sur les optimisations possibles. Ainsi, une manière simple de représenter ce 
graphe est de considérer l'ensemble des couples des étiquettes de blocs qui indiqueront un arc du premier élément au second. 
Pour ce faire, considérons l'application suivante, 

\begin{align*}
	\varrho : Stm &\longrightarrow \mathcal{P}(\mathbb{N}^2) \\
	\sassign{Id}{Exp_a} &\longmapsto \emptyset \\
	\sskip &\longmapsto \emptyset \\
	\sseq{s_1}{s_2} &\longmapsto \varrho(s_1) \cup \varrho(s_2) \cup [\phi(s_1)\times\{\iota(s_2)\}] \\
	\sifthenelse{b^l}{s_1}{s_2} &\longmapsto \varrho(s_1)\cup\varrho(s_2)\cup(l, \iota(s_1))\cup(l, \iota(s_2)) \\
	\swhiledo{b^l}{s} &\longmapsto \varrho(s)\cup(l,\iota(s))\cup[\phi(s)\times\{l\}]
\end{align*}
On introduira aussi un accès à l'ensemble des successeurs d'un bloc par la fonction,
\begin{align*}
	succ : \mathbb{N} &\longrightarrow \mathcal{P}(\mathbb{N})\\
	l &\longmapsto \{l' \in \mathbb{N} \mid (l, l') \in \mathcal{G}_V\},
\end{align*}
où $\mathcal{G}_V$ est l'ensemble des arcs du graphe de flot de contrôle du programme.

\section{Analyse de vivacité}
La première analyse sur laquelle nous travaillons est l'analyse de vivacité des variables de nos programmes. 
Il s'agira donc de déterminer pour chaque bloc d'un programme, l'ensemble de variables encore vivantes, c'est-à-dire
encore utilisées une fois le bloc courrant passé. Pour ce faire, il faut au préalable déclarer quelques ensembles 
nécessaires à cette analyse.

\subsection{Description ensembliste du programme}
Soit $V$ l'ensemble des variables déclarées dans un programme. On se donne alors $S = (\mathcal{P}(V), \subseteq)$
l'ensemble partiellement ordonné des parties de $V$.
\\
Pour $l \in \mathbb{N}$ l'étiquette d'un bloc, on définit désormais les ensembles qui suivent,
\[GEN[l] \subseteq \mathcal{P}(V)\]
l'ensemble des variables appelées dans le bloc en question et,
\[KILL[l] \subseteq \mathcal{P}(V)\]
l'ensemble des variables nouvellement assignées donc considérées pour lors comme mortes. Enfin on se donnera,
\[vars : Exp_a \cup Exp_b \longrightarrow \mathcal{P}(V)\]
l'ensemble des identifiants présents dans une expression arithmétique ou booléenne.
\\
On peut désormais introduire les fonctions,
\begin{align*}
	gen : Block &\longrightarrow \mathcal{P}(V)\\
	\sassign{Id}{Exp_a} &\longmapsto vars(Exp_a)\\
	\sskip &\longmapsto \emptyset\\
	Exp_b &\longmapsto vars(Exp_b)
\end{align*}
génère l'ensemble $GEN[l]$ et,
\begin{align*}
	kill : Block &\longrightarrow \mathcal{P}(V)\\
	\sassign{Id}{Exp_a} &\longmapsto \{Id\}\\
	\sskip &\longmapsto \emptyset\\
	Exp_b &\longmapsto \emptyset
\end{align*}
génère l'ensemble $KILL[l]$.
\\
\\
Notre langage étant un langage impératif, il est par conséquent naturel d'effectuer cette analyse de bas en haut dans le programme,
de manière à déterminer cette vivacité. Ainsi l'on déclare deux ensembles de variables vivantes à l'entrée, et à la
sortie d'un bloc. On les définit comme tels,

\[
	LIVE_{out}[l] = 
	\begin{dcases*}
		\emptyset &si $succ(l) = \emptyset $\,, \\
		\bigcup\limits_{p\in succ(s)} LIVE_{in}[p] &sinon\,,
	\end{dcases*}
\]
l'ensemble des variables vivantes à la sortie d'un bloc et,

\[
	LIVE_{in}[l] = GEN[l] \cup (LIVE_{out}[l] - KILL[l]),
\]
l'ensemble des variables vivantes à l'entrée d'un bloc.
\\
\\
En tant que tel, nous ne pouvons toujours pas effectuer l'analyse de vivacité sur nos programmes, il nous manque en
effet un algorithme de point fixe pour y parvenir. Dans les parties suivantes, nous introduirons pour ce faire, 
l'algorithme du point fixe de Kleene.

\subsection{Monotonie}
La monotonie des ensembles de flot de données nous permet de déterminer l'existence d'un point fixe sur notre
fonction. Cela sera donc utile pour fournir un algorithme qui termine, lors de la construction de ces ensembles, 
étant donné que l'ensemble des variables d'un programme est supposé fini.
\\
\\
On se consacre, dans cette partie, à montrer la monotonie de nos ensembles $LIVE_{in}[l]$ et $LIVE_{out}[l]$
pour toute étiquette $l$ de blocs. Posons la fonction,
\begin{align*}
	f_l : \mathcal{P}(V) &\longrightarrow \mathcal{P}(V) \\
	\mathcal{O} &\longmapsto GEN[l] \cup (\mathcal{O} - KILL[l])
\end{align*}
Soient $k, k' \in \mathcal{P}(V)$ tels que $k \subseteq k'$, alors on a,
\begin{align*}
	k - KILL[l] &\subseteq k' - KILL[l] \text{ et,}\\
	GEN[l] \cup (k - KILL[l]) &\subseteq GEN[l] \cup (k' - KILL[l]).
\end{align*}
Donc $f_l(k) \subseteq f_l(k')$ ce qui implique $f_l$ monotone. Comme $LIVE_{out}[l]$ est une union de tous les $LIVE_{in}$
de ses successeurs et que $f_l$ est une fonction monotone, on a que $LIVE_{out}[l]$ est monotone pour tout $l$, par récurrence
sur la monotonie de l'union. Il vient de plus que $LIVE_{in}[l]$, dépendant de $LIVE_{out}[l]$, est monotone.
Ainsi ces deux fonctions dépendants de l'étiquette d'un bloc sont monotones et produisent des ensembles finis, étant donné
la finitude de $V$ dans notre cas.

\subsection{Point fixe}
Revenons-en à l'existence d'un point fixe, pour pouvoir construire un algorithme itératif. En effet,
on a que $(\mathcal{P}(V), \subseteq)$ l'ensemble de nos variables est un ordre partiel et est fini.
Ainsi, grâce à la monotonie de nos deux ensembles $LIVE_{in}$ et $LIVE_{out}$ démontrée ci-dessus, il
vient qu'à chaque itération de notre algorithme, l'ensemble produit sera soit identique à l'ensemble précédent,
soit plus gros et la finitude de l'ensemble des variables assure qu'il ne pourra par grossir indéfiniment. On se sert pour ce faire
du théorème du point fixe de Kleene qui s'énnonce comme suit,
\newline
\newline
\theorem{
	Soit $(L, \sqsubseteq)$ un ordre partiellement ordonné, avec un plus petit élément  $\perp$ et soit
	une application $f : L \longrightarrow L$ monotone. Alors il existe un point fixe minimal qui est le suprémum de la suite,
	\[\perp \sqsubseteq f(\perp) \sqsubseteq f^2(\perp) \sqsubseteq \cdots \sqsubseteq f^k(\perp) \sqsubseteq \cdots\]
}
\begin{algorithm}
	\caption{Itération du point fixe}
	\begin{algorithmic}
		\State {Soit $S$ le programme}
		\State $\mathcal{B} \leftarrow \beta(S)$
		\State {Soit $L$ l'ensemble des étiquettes de $\mathcal{B}$}
		\For{$l \in L$}
		\State $live_{in}[l] \leftarrow \emptyset$
		\State $live_{out}[l] \leftarrow \emptyset$
		\EndFor
		\While{$live_{in} \ne live_{in}' \textbf{ ou } live_{out} \ne live_{out}'$}
		\For{$l \in L$}
		\State $live_{out}[l] \leftarrow \bigcup\limits_{p\in succ(l)} live_{in}[p]$
		\State $live_{in}[l] \leftarrow gen[l] \cup (live_{out}[l] - kill[l])$
		\EndFor
		\EndWhile
	\end{algorithmic}
\end{algorithm}

Si $n = \#\mathcal{B}$, alors cet algorithme a une complexité en $O(kn)$, où $k$ est la hauteur du diagramme
de Hasse sur $(\mathcal{P}(V), \subseteq)$. Cela se montre grâce à la monotonie de $LIVE_{in}$, en effet,
celle-ci ne fait que croître vers $\top$.
Concernant l'implémentation, il peut être intéressant de définir $live$ comme une structure binaire de taille au
moins $m$ bits avec $m = \#V$.
Une amélioration peut être simplement faite sur l'algorithme décrit ci-dessus. Effectivement, on peut remarquer que,
à chaque modification de $LIVE_{out}[l]$ seul les blocs successeurs seront susceptibles de s'altérer. Donc
au lieu d'itérer à nouveau sur l'ensemble des blocs, itérer uniquement sur les blocs successeurs au dernier
bloc altéré. Cette approche a pour nom naturel de \textit{worklist}.
\\
\\
La correction de l'algorithme est la suivante, elle repose en partie sur ce qui a été dit en amont de cette section.
En effet, à chaque itération, dans laquelle l'on vérifie si le point fixe a ou non été atteint, on met à jour les
tableaux $live_{in}[l]$ et $live_{out}[l]$ pour tout bloc. Mais comme pour ce faire, on utilise deux fonctions monotones croissantes et qu'on suppose
l'ensemble de nos variables fini, on est assuré de trouver le plus petit point fixe existant (par le théorème de Kleene) et ainsi de terminer l'algorithme.

\end{document}