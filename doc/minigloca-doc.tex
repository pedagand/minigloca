\documentclass[a4paper, 12pt]{article}

\usepackage[francais]{babel}
\usepackage[T1]{fontenc}
\usepackage{amssymb}
\usepackage{amsmath}
\usepackage{mathpartir}
\usepackage{tikz}
\usepackage{listings}
\usepackage{algorithm}
\usepackage{algpseudocode}
\usepackage{color}
\usepackage{mathtools} % for 'dcases*' env.

\definecolor{dkgreen}{rgb}{0,0.6,0}
\definecolor{gray}{rgb}{0.5,0.5,0.5}
\definecolor{mauve}{rgb}{0.58,0,0.82}

\floatname{algorithm}{Algorithme}

\lstset{frame=tb,
	language=caml,
	columns=[c]fixed,
	basicstyle=\small\ttfamily,
	keywordstyle=\bfseries,
	upquote=true,
	commentstyle=,
	breaklines=true,
	showstringspaces=false,
	stringstyle=\color{green}
}

\title{Minigloca}
\author{Vladislas de Haldat}

\begin{document}

\newcommand{\deftype}[2]{
    ${#1}\rightarrow$\foreach[count=\i] \e in {#2} {
        \ifnum\i>1$\mid$\fi
        \space${\e}$\\
    }
}

\newcommand{\definition}[1] {
    \textbf{\textit{Définition --\space}}{#1}
}

\newcommand{\theorem}[1] {
    \textbf{\textit{Théorème --\space}}{#1}
}

\newcommand{\sassign}[2] {
    {#1}\space:=\space{#2}
}

\newcommand{\sseq}[2] {
    {#1}\space;\space{#2}
}

\newcommand{\sskip}{
    \textit{Skip}
}

\newcommand{\ifthenelse}[3] {
    \text{if }{#1}\text{ then }{#2}\text{ else }{#3}
}

\newcommand{\whiledo}[2] {
    \text{while }{#1}\text{ do }{#2}
}

\maketitle
\newpage
\tableofcontents
\newpage

\section{Un langage impératif simple}
Pour commencer, définissons une syntaxe abstraite minimale que nous utiliserons tout le long de cette étude.
Cette syntaxe se composera de trois blocs fondamentaux que sont les expressions arithmétiques, les expressions
booléennes ainsi que les déclarations. Nous n'incluons pas pour le moment la déclaration de routines au sein de cette syntaxe.

\subsection{Expressions arithmétiques}
Les expressions arithmétiques sont définies sur l'ensemble des entiers relatifs. On se donne les opérateurs
de l'addition, de la soustraction ainsi que de la multiplication. À ces opérateurs l'on pourra appliquer des
entiers ainsi que des identifiants de variables.

% \begin{dtype}{Int}
%   \inlinekind{n}\with{n\in\mathbb{Z}}
% \end{dtype}
% \begin{dtype}{Id}
%   \inlinekind{x}
%   \kind{y}
%   \kind{z}
%   \kind{\ldots}
% \end{dtype}
% \begin{dtype}{Exp_a}
%   \inlinekind{Int}
%   \kind{Id}
%   \kind{op_A (a_1, a_2)}
%   \with{op_A \in\{+, -, \times\}}
% \end{dtype}

\begin{align*}
  Int   & ::= n                               & n \in \mathbb{Z}          \\
  Id    & ::= x \mid y \mid z \mid \cdots                                 \\
  Exp_a & ::= n \mid x \mid op_A(a_1, a_2) & op_A \in \{+, -, \times\}
\end{align*}

\subsection{Expressions booléennes}
Les expressions booléennes nous permettent d'introduire la comparaison entre deux expressions arithmétiques,
ainsi que les opérateurs booléens sur les expressions booléennes.

\begin{dtype}{b}
  \inlinekind{\textbf{true}}
  \kind{\textbf{false}}\\
  \akind{op_R (a_1, a_2)}\with{op_R\in\{<, =\}}
  \akind{op_B (b_1, b_2)}\with{op_B\in\{\wedge, \vee\}}
  \akind{\neg{b}}
\end{dtype}

\subsection{Déclarations}
Les déclarations sont définies de la manière suivante:

\begin{dtype}{s}
  \inlinekind{\sassign{x}{a}}\\
  \akind{\sseq{s_1}{s_2}}\\
  \akind{\sskip}\\
  \akind{\sifthenelse{b}{s_1}{s_2}}\\
  \akind{\swhiledo{b}{s}}
\end{dtype}

\subsection{Exemple}
\noindent
Voici un premier exemple sur ce langage
\begin{lstlisting}[tabsize=2]
a := 1;
b := 20;

if a = 3 then
	c := 4
else
	c := 6
endif;

while b < 100 do
	b := b + 1
done
\end{lstlisting}

\subsection{Arbre de syntaxe abstraite}
Ce modèle de représentation permet de construire un arbre de syntaxe abstraite (AST) du programme en question,
cette structure étant un moyen pratique de le représenter. Cela nous servira particulièrement lors des différentes
analyses qui seront effectuées, et notamment l'analyse par flot de contrôle. Cette dernière ne requièrant pas
la connaissance de l'ordre d'exécution du programme, les AST sont tout-à-fait adaptés!

\begin{center}\input{ast.fig}\end{center}

\section{Interpréteur et sémantique}
Dans ce chapitre, on s'atèle à décrire l'interpréteur ainsi que la sémantique sur notre petit langage impératif. 
Dans ce cadre là, on définit l'état du programme par une bijection entre les identifiants des variables et leur valeur, 
ici un entier:
\[\sigma : \mathbb{V} \longrightarrow \mathbb{Z}\]
On introduire aussi la bijection $\mu_B$,
\[\mu_B : \mathcal{B} \longrightarrow \mathbb{B}\]
où $\mathcal{B} = \{\textbf{true}$, \textbf{false}\} les valeurs booléennes de notre langage et $\mathbb{B} = \{\top, \bot\}$ 
l'ensemble des valeurs booléennes natives à OCaml. Pour la suite, on considèrera l'ensemble des états $\mathcal{S} = \mathbb{Z}^\mathbb{V}$.
Les états décriront l'évolution de l'exécution du programme.

\subsection{Interpréteur}
Sur les expressions arithmétiques, définies dans le section 1, on définit la fonction suivante:
\begin{align*}
	\intrfun{a}{A}:\mathcal{S}&\longrightarrow\mathbb{Z}\\
	\intrfun{x}{A}\sigma&\longmapsto\sigma(x)\\
	\intrfun{op_A(a_1, a_2)}{A}\sigma&\longmapsto\hat{op}_A(\intr{a_1}{A}{\sigma}\text{, }\intr{a_2}{A}{\sigma})\\
\end{align*}
Cette dernière est aussi définie respectivement aux autres opérateurs arithmétiques, introduits dans le chapitre précédent.
De la même manière, on définit la fonction qui suit sur les expressions booléennes:
\begin{align*}
	\intrfun{b}{B}:\mathcal{S}&\longrightarrow\mathbb{B}\\	
	\intrfun{\textbf{true}}{B}\sigma&\longmapsto\top\\
	\intrfun{\textbf{false}}{B}\sigma&\longmapsto\bot\\
	\intrfun{op_R(a_1, a_2)}{B}\sigma &\longmapsto \hat{op}_R(\intr{a_1}{A}{\sigma}\text{, }\intr{a_2}{A}{\sigma})\\
	\intrfun{op_B(b_1, b_2)}{B}\sigma&\longmapsto \hat{op}_B(\intr{b_1}{B}{\sigma}\text{, }\intr{b_2}{B}{\sigma})\\
	\intrfun{\neg b}{B}\sigma&\longmapsto\hat{\neg}\intr{b}{B}{\sigma}
\end{align*}
Les opérateurs de la forme $\hat{op}$ représentent les opérateurs natifs à OCaml.

\subsection{Sémantique}
Maintenant que nous avons correctement défini l'interpréteur, il est possible 
de construire la sémantique du langage. Cela permet également de développer 
un ensemble de règles logiques. La syntaxe de la déclaration des règles,
prenant en compte nos état et déclaration, sera donc de la forme suivante,

\[Stm, \mathcal{S} \longrightarrow Stm', \mathcal{S}'\]
\\
où $Stm$ est la première déclaration, $Stm'$ celle qui suit après son exécution, 
$\mathcal{S}$ l'état initial et $\mathcal{S}'$  l'état successeur. Commençons 
par la déclaration vide,

\srule{ }{\semanticd{\sskip}{\sigma}{\emptyset}{\sigma}}

Poursuivons avec la déclaration de l'affectation,
\srule{ }{\semanticd{\sassign{x}{a}}{\sigma}{\emptyset}{\sigma'[Id\longmapsto\intr{a}{A}{\sigma}]}}
Ici, le nouvel état $\sigma'$ lie l'identifiant de la variable assignée, à la 
valeur de l'expression arithmétique $\intrfun{a}{A}$. Poursuivons avec la règle 
de la séquence entre deux déclarations, d'une part si la première termine,
\srule{\semanticd{Stm_1}{\sigma}{\emptyset}{\sigma'}}
{\semanticd{\sseq{Stm_1}{Stm_2}}{\sigma}{Stm_2}{\sigma'}}
D'autre part, si la première ne termine pas,
\srule{\semanticd{Stm_1}{\sigma}{Stm_1'}{\sigma'}}
{\semanticd{\sseq{Stm_1}{Stm_2}}{\sigma}{\sseq{Stm_1'}{Stm_2}}{\sigma'}}
La condition peut se formaliser de la sorte dans le cas où la garde est vérifiée,
\srule{\intr{b}{B}{\sigma}}
{\semanticd{\sifthenelse{b}{Stm_1}{Stm_2}}{\sigma}{Stm_1}{\sigma}}
Dans le cas où elle ne l'est pas,
\srule{\neg\intr{b}{B}{\sigma}}
{\semanticd{\sifthenelse{b}{Stm_1}{Stm_2}}{\sigma}{Stm_2}{\sigma}}
On peut déclarer la règle qui suit pour la déclaration while,
\srule{\intr{b}{B}{\sigma}}
{\semanticd{\swhiledo{b}{Stm}}{\sigma}{\sseq{Stm}{\swhiledo{b}{Stm}}}{\sigma}}

\section{Prérequis à l'analyse}
De manière à pouvoir travailler avec les analyses de flot de données et de flot de contrôle, 
il est nécessaire d'approfondir le principe d'étiquettage déjà abordé à la fin du premier section.

\subsection{Étiquettage}
Avant tout, introduisons la fonction suivante,

\[
	\lambda: s \longrightarrow \mathbb{L}
\]
\newline
qui prend une déclaration $s$ et retourne une étiquette unique.
\newline
\newline
\definition{Soient $s \in Stm$ et $(l_n)_{n\in\mathbb{N}}$ une liste d'étiquettes, $s$ est dite bien formée si et seulement}
si $\forall i \in \mathbb{N}, \forall j \in \mathbb{N}$ tel que $i \ne j$ alors $l_i \ne l_j$.
\newline
\newline
Étant donné une déclaration $S$ définie par l'ensemble de ses déclarations atomiques, on a alors que 
$\pi \in \mathcal{P}(s)$, un sous bloc de ce programme, n'admet qu'une seule entrée et une ou plusieurs sorties. 
On peut maintenant définir la fonction $\iota$, qui retournera la première étiquette rencontrée dans une déclaration,

\begin{align*}
	\iota : Stm &\longrightarrow \mathbb{N}\\
	(\sassign{Id}{Exp_a})^l &\longmapsto l\\
	\sseq{s_1}{s_2} &\longmapsto \iota(s_1)\\
	{\sskip}^l &\longmapsto l\\
	\sifthenelse{b^l}{s_1}{s_2} &\longmapsto l\\
	\swhiledo{b^l}{s} &\longmapsto l
\end{align*}
Comme expliqué précédemment, il est aussi nécessaire de déclarer une fonction $\phi$ qui retournera l'ensemble des étiquettes finales 
à la fin d'un bloc $\pi$.

\begin{align*}
	\phi : Stm &\longrightarrow \mathcal{P}(\mathbb{N})\\
	(\sassign{Id}{Exp_a})^l &\longmapsto \{l\}\\
	\sseq{s_1}{s_2} &\longmapsto \phi(s_2)\\
	\sskip^l &\longmapsto \{l\}\\
	\sifthenelse{b^l}{s_1}{s_2} &\longmapsto \phi(s_1) \cup \phi(s_2)\\
	\swhiledo{b^l}{s} &\longmapsto \{l\}
\end{align*}
\subsection{Blocs}
De manière à faciliter l'analyse d'un programme, il est utile de partitionner et de factoriser notre représentation du code. 
Les blocs ont été évoqués dans la partie précédente comme des sous-parties de l'ensemble des déclarations atomiques d'un programme. 
On peut désormais les définir de la sorte,

\begin{dtype}{Block}
	\inlinekind{\sassign{Id}{Exp_a}}\\
	\akind{Exp_b}\\
	\akind{\sskip}
\end{dtype}
Maintenant les blocs correctement définis, il nous faut pouvoir les lier à une étiquette. Pour ce faire, on se donne $\beta$ définie par,

\begin{align*}
	\beta : Stm &\longrightarrow \mathcal{P}(\mathbb{N} \times Block)\\
	(\sassign{Id}{Exp_a})^l&\longmapsto\{(l, Id := Exp_a)\}\\
	\sseq{s_1}{s_2} &\longmapsto \beta(s_1) \cup \beta(s_2)\\
	\sskip^l &\longmapsto \{(l, \textit{Skip})\}\\
	\sifthenelse{b^l}{s_1}{s_2} &\longmapsto \{(l, b)\}\cup \beta(s_1)\cup\beta(s_2)\\
	\swhiledo{b^l}{s} &\longmapsto \{(l, b)\}\cup\beta(s)
\end{align*}
À partir de là, il est possible de formaliser les graphes orientés de flot. Les blocs en représentent les noeuds, et le passage 
vers le bloc suivant est représenté par un arc.

\subsection{Flots}
Un graphe de flot de contrôle est un graphe orienté dont les noeuds représentent une déclaration et les arcs un
flot de contrôle. Chaque noeud est une unique opération de notre programme. Dans notre cas, le graphe de flot de
contrôle aura un seul point d'entrée mais plusieurs points de sortie. Ils sont considérés comme des noeuds ``sans opération''.
De manière à construire ce graphe, on peut assigner à chaques déclarations procurant une unique exécution, une étiquettes.
Dans notre cas, ces déclarations sont l'affectation, la condition ainsi que la boucle. En voici quelques exemples:

\begin{center}\input{cfg.fig}\end{center}
Nous avons désormais toutes les structures nécéssaires à la construction de notre graphe de flot. Pour le moment, il s'agira de 
le construire naïvement, l'on reviendra plus tard sur les optimisations possibles. Ainsi, une manière simple de représenter ce 
graphe est de considérer l'ensemble des couples des étiquettes de blocs qui indiqueront un arc du premier élément au second. 
Pour ce faire, considérons l'application suivante, 

\begin{align*}
	\varrho : s &\longrightarrow \mathcal{P}(\mathbb{L}^2) \\
	\sassign{x}{a} &\longmapsto \emptyset \\
	\sskip &\longmapsto \emptyset \\
	\sseq{s_1}{s_2} &\longmapsto \varrho(s_1) \cup \varrho(s_2) \cup [\phi(s_1)\times\{\iota(s_2)\}] \\
	\sifthenelse{b^l}{s_1}{s_2} &\longmapsto \varrho(s_1)\cup\varrho(s_2)\cup(l, \iota(s_1))\cup(l, \iota(s_2)) \\
	\swhiledo{b^l}{s} &\longmapsto \varrho(s)\cup(l,\iota(s))\cup[\phi(s)\times\{l\}]
\end{align*}
On introduira aussi un accès à l'ensemble des successeurs d'un bloc par la fonction,
\begin{align*}
	succ : \mathbb{N} &\longrightarrow \mathcal{P}(\mathbb{N})\\
	l &\longmapsto \{l' \in \mathbb{N} \mid (l, l') \in \mathcal{G}_V\},
\end{align*}
où $\mathcal{G}_V$ est l'ensemble des arcs du graphe de flot de contrôle du programme.

\section{Analyse de vivacité}
La première analyse sur laquelle nous travaillons est l'analyse de vivacité des variables de nos programmes. 
Il s'agira donc de déterminer pour chaque bloc d'un programme, l'ensemble de variables encore vivantes, c'est-à-dire
encore utilisées une fois le bloc courrant passé. Pour ce faire, il faut au préalable déclarer quelques ensembles 
nécessaires à cette analyse.

\subsection{Description ensembliste du programme}
Soit $\mathbb{V}$ l'ensemble des variables déclarées dans un programme. On se donne alors $S = (\mathcal{P}(\mathbb{V}), \subseteq)$
l'ensemble partiellement ordonné des parties de $\mathbb{V}$.
\\
Pour $l \in \mathbb{L}$ l'étiquette d'un bloc, on définit désormais les ensembles qui suivent,
\[\sgen{l} \subseteq \mathcal{P}(\mathbb{V})\]
l'ensemble des variables appelées dans le bloc en question et,
\[\skill{l} \subseteq \mathcal{P}(\mathbb{V})\]
l'ensemble des variables nouvellement assignées donc considérées pour lors comme mortes. Enfin on se donnera,
\[vars : s \longrightarrow \mathcal{P}(\mathbb{V})\]
l'ensemble des identifiants présents dans une expression arithmétique ou booléenne.
\\
On peut désormais introduire les fonctions,
\begin{align*}
	gen : Block &\longrightarrow \mathcal{P}(\mathbb{V})\\
	\sassign{x}{a} &\longmapsto vars(a)\\
	\sskip &\longmapsto \emptyset\\
	b &\longmapsto vars(b)
\end{align*}
génère l'ensemble $\sgen{l}$ et,
\begin{align*}
	kill : Block &\longrightarrow \mathcal{P}(\mathbb{V})\\
	\sassign{x}{a} &\longmapsto \{x\}\\
	\sskip &\longmapsto \emptyset\\
	b &\longmapsto \emptyset
\end{align*}
génère l'ensemble $\skill{l}$.
\\
\\
Notre langage étant un langage impératif, il est par conséquent naturel d'effectuer cette analyse de bas en haut dans le programme,
de manière à déterminer cette vivacité. Ainsi l'on déclare deux ensembles de variables vivantes à l'entrée, et à la
sortie d'un bloc. On les définit comme tels,

\[
	LIVE_{out}[l] = 
	\begin{dcases*}
		\emptyset &si $succ(l) = \emptyset $\,, \\
		\bigcup\limits_{p\in succ(s)} LIVE_{in}[p] &sinon\,,
	\end{dcases*}
\]
l'ensemble des variables vivantes à la sortie d'un bloc et,

\[
	LIVE_{in}[l] = \sgen{l} \cup (LIVE_{out}[l] - \skill{l}),
\]
l'ensemble des variables vivantes à l'entrée d'un bloc.
\\
\\
En tant que tel, nous ne pouvons toujours pas effectuer l'analyse de vivacité sur nos programmes, il nous manque en
effet un algorithme de point fixe pour y parvenir. Dans les parties suivantes, nous introduirons pour ce faire, 
l'algorithme du point fixe de Kleene.

\subsection{Monotonie}
La monotonie des ensembles de flot de données nous permet de déterminer l'existence d'un point fixe sur notre
fonction. Cela sera donc utile pour fournir un algorithme qui termine, lors de la construction de ces ensembles, 
étant donné que l'ensemble des variables d'un programme est supposé fini.
\\
\\
\begin{lemma}
	$\livein{\cdot}$ et $\liveout{\cdot}$ sont monotones.
\end{lemma}
\\
\begin{proof}	
Posons l'application pour $l \in \mathbb{N}$,
\begin{align*}
	f_l : \mathcal{P}(\mathbb{V}) &\longrightarrow \mathcal{P}(\mathbb{V}) \\
	\mathcal{O} &\longmapsto \sgen{l} \cup (\mathcal{O} - \skill{l})
\end{align*}
Soient $K, K' \in \mathcal{P}(\mathbb{V})$ tels que $K \subseteq K'$, alors on a,
\begin{align*}
	K - \skill{l} &\subseteq K' - \skill{l} \text{ et,}\\
	\sgen{l} \cup (K - \skill{l}) &\subseteq \sgen{l} \cup (K' - \skill{l}).
\end{align*}
Donc $f_l(k) \subseteq f_l(k')$ ce qui implique $f_l$ monotone. Comme $\liveout{l}$ est une union de tous les $\livein{\cdot}$
de ses successeurs et que $f_l$ est une fonction monotone, on a que $\liveout{l}$ est monotone pour tout $l$, par monotonie 
de l'union. Il vient de plus que $\livein{l}$, composé avec $\liveout{l}$, est monotone.
Ainsi ces deux fonctions dépendants de l'étiquette d'un bloc sont monotones et produisent des ensembles finis, étant donné
la finitude de $\mathbb{V}$ dans notre cas.
\hfill$\square$
\end{proof}

\subsection{Point fixe}
Revenons-en à l'existence d'un point fixe, pour pouvoir construire un algorithme itératif. En effet,
on a que $(\mathcal{P}(V), \subseteq)$ l'ensemble de nos variables est un ordre partiel et est fini.
Ainsi, grâce à la monotonie de nos deux ensembles $\livein{\cdot}$ et $\liveout{\cdot}$ démontrée ci-dessus, il
vient qu'à chaque itération de notre algorithme, l'ensemble produit sera soit identique à l'ensemble précédent,
soit plus gros et la finitude de l'ensemble des variables assure qu'il ne pourra par grossir indéfiniment. On se sert pour ce faire
du théorème du point fixe de Kleene qui s'énnonce comme suit,
\newline
\newline
\begin{theorem}	
	Soit $(L, \sqsubseteq)$ un ordre partiellement ordonné, avec un plus petit élément  $\perp$ et soit
	une application $f : L \longrightarrow L$ monotone. Alors il existe un point fixe minimal qui est le suprémum de la suite,
	\[\perp \sqsubseteq f(\perp) \sqsubseteq f^2(\perp) \sqsubseteq \cdots \sqsubseteq f^k(\perp) \sqsubseteq \cdots\]
\end{theorem}

\begin{algorithm}
	\caption{Itération du point fixe}
	\begin{algorithmic}
		\State {Soit $S$ le programme}
		\State $\mathcal{B} \leftarrow \beta(S)$
		\State {Soit $L$ l'ensemble des étiquettes de $\mathcal{B}$}
		\For{$l \in L$}
		\State $live_{in}[l] \leftarrow \emptyset$
		\State $live_{out}[l] \leftarrow \emptyset$
		\EndFor
		\While{$live_{in} \ne live_{in}' \textbf{ ou } live_{out} \ne live_{out}'$}
		\For{$l \in L$}
		\State $live_{out}[l] \leftarrow \bigcup\limits_{p\in succ(l)} live_{in}[p]$
		\State $live_{in}[l] \leftarrow \sgen[l] \cup (live_{out}[l] - \skill{l})$
		\EndFor
		\EndWhile
	\end{algorithmic}
\end{algorithm}

Si $n = \#\mathcal{B}$, alors cet algorithme a une complexité en $O(kn)$, où $k$ est la hauteur du diagramme
de Hasse sur $(\mathcal{P}(V), \subseteq)$. Cela se montre grâce à la monotonie de $\livein{\cdot}$, en effet,
celle-ci ne fait que croître vers $\top$.
Concernant l'implémentation, il peut être intéressant de définir $live$ comme une structure binaire de taille au
moins $m$ bits avec $m = \#V$.
Une amélioration peut être simplement faite sur l'algorithme décrit ci-dessus. Effectivement, on peut remarquer que,
à chaque modification de $\liveout{\cdot}$ seuls les blocs prédecesseurs seront susceptibles de s'altérer. Donc
au lieu d'itérer à nouveau sur l'ensemble des blocs, itérer uniquement sur les blocs successeurs au dernier
bloc altéré. Cette approche a pour nom naturel de \textit{worklist}.
\\
\\
\begin{lemma}
	Étant donné un programme $P$, à partir de $\bot$ l'algorithme d'itération du point fixe trouve effectivement
	un point fixe sur $P$ et termine.
\end{lemma}
\\
\begin{proof}
La correction de l'algorithme est la suivante, elle repose en partie sur ce qui a été dit en amont de cette section.
En effet, à chaque itération, dans laquelle l'on vérifie si le point fixe a ou non été atteint, on met à jour les
tableaux $live_{in}[l]$ et $live_{out}[l]$ pour tout bloc. Mais comme pour ce faire, on utilise deux fonctions monotones croissantes et qu'on suppose
l'ensemble de nos variables fini, on est assuré de trouver le plus petit point fixe existant (par le théorème de Kleene) et ainsi de terminer l'algorithme.
\hfill$\square$
\end{proof}
\\
Dans cette étude, nous travaillerons avec les deux algorithmes évoqués pour réaliser les tests unitaires. Ci-dessous, un aperçu du fonctionnement
de l'algorithme par worklist.

\begin{algorithm}
	\caption{Itération du point fixe (worklist)}
	\begin{algorithmic}
		\State {Soit $S$ le programme}
		\State {Soit $L$ l'ensemble des étiquettes de $\mathcal{B}$}
		\State $\mathcal{Q} \leftarrow L$ une queue des étiquettes
		\For{$l \in L$}
		\State $live_{in}[l] \leftarrow \emptyset$
		\State $live_{out}[l] \leftarrow \emptyset$
		\EndFor
		\While{$\#\mathcal{Q} > 0$}
		\State $q \leftarrow \mathcal{Q}$.pop
		\State $live_{out}[q] \leftarrow \bigcup\limits_{p\in succ(q)} live_{in}[p]$
		\State $live_{in}[q] \leftarrow \sgen{q} \cup (live_{out}[q] - skill{q})$
		\If {$live_{in}[q] \ne live_{in}'[q]$}
		\State $\mathcal{Q}$.push(pred(q))
		\EndIf
		\EndWhile
	\end{algorithmic}
\end{algorithm}
\noindent
\begin{lemma}
	Étant donné un programme $P$, à partir de $\bot$ l'algorithme par worklist trouve effectivement
	un point fixe sur $P$ et termine.
\end{lemma}
\\
\begin{proof}
On remarque que, à la modification de l'analyse de vivacité $\liveout{\cdot}$ à la sortie d'un bloc,
les blocs successeurs ne seront nullement impactés. En effet seuls les blocs prédecesseurs seront
altérés d'après la définition. L'algorithme est donc une restriction de l'algorithme naïf vu plus
haut. À chaque modification effective, on ajoute à la queue des blocs à traiter, l'ensemble des blocs
prédecesseurs au bloc en question. Ensuite, par monotonie de $\livein{\cdot}$ et $\liveout{\cdot}$ et
par finitude de l'ensemble de nos variables, on est assuré de trouver le plus petit point fixe existant
et ainsi terminer l'algorithme.	
\end{proof}
\subsection{Exemple}
\noindent
Considérons un programme simple comme celui-ci,
\begin{lstlisting}[tabsize=2]
	a := 0;
	b := a;
	while a < 100 do
		a := a + 1
	done;
	c := b + a
\end{lstlisting}
On obtient donc l'analyse de vivacité suivante, en appliquant l'un des deux algorithmes explicités ci-dessus,
\\
\begin{center}
	\begin{tabular}{||c|l|r|l||}
	\hline
	Étiq. & Bloc & $\livein{\cdot}$ & $\liveout{\cdot}$ \\
	\hline
	1 & a := 0 & $\emptyset$ & $\{a\}$\\
	2 & b := a & $\{a\}$ & $\{a, b\}$\\
	3 & while a < 100 do & $\{a, b\}$ & $\{a, b\}$\\
	4 & a := a + 1 & $\{a, b\}$ & $\{a, b\}$\\
	- & done & - & -\\
	5 & c := b + a & $\{a, b\}$ & $\emptyset$\\
	\hline
	\end{tabular}
\end{center}
\section{Élimination de code mort}
Cette analyse permet une première optimisation qu'est l'élimination de code mort. Elle consiste, en pré-compilation,
à générer un nouveau code à partir du code initial, dans lequel le code dit mort n'est plus présent.
On définira ici le code mort comme un ensemble de blocs d'un programme qui ne seront jamais utilisés.

\subsection{Réduction naïve}
Cette optimisation peut en premier lieu être abordée de manière naïve. On notera $A$ l'analyse de flot de donnée.
Considérons $\Delta$ la fonction d'élimination du code mort définie comme:
\[\Delta : \mathcal{A} \times Stm \longrightarrow Stm\]
qui, à partir du code d'un programme et d'une analyse de flot de donnée, produit un nouveau code. $\Delta$ est itérée
jusqu'à atteindre un point fixe sur le code envoyé, et le code reçu.

\subsubsection{Exemple}
\noindent
Considérons le programme suivant.
\begin{lstlisting}[tabsize=2]
	a := 1;
	b := 0;
	c := a + b;
	while b < 100 do
		b := b + 1
	done;
	d := 2 * c
\end{lstlisting}
On a donc l'analyse de vivacité suivante,
\begin{center}
\begin{tabular}{||c|l|r|l||}
\hline
Étiq. & Bloc & $\livein{\cdot}$ & $\liveout{\cdot}$ \\
\hline
1 & a := 1 & $\emptyset$ & $\{a\}$\\
2 & b := 20 & $\{a\}$ & $\{a, b\}$\\
3 & c := a + b & $\{a, b\}$ & $\{b, c\}$\\
4 & while b < 100 do & $\{b, c\}$ & $\{b, c\}$\\
5 & b := b + 1 & $\{b, c\}$ & $\{b, c\}$\\
- & done & - & -\\
6 & d := 2 * c & $\{c\}$ & $\emptyset$\\
\hline
\end{tabular}
\end{center}
On remarque que $\liveout{6} = \emptyset$ donc $d$ est morte après avoir été assignée. L'algorithme applique donc sur ce bloc
une tranformation vers une instruction $Skip$, puis calcule à nouveau l'analyse, à partir de $\bot$. On obtient donc l'analyse
de vivacité,
\begin{center}
	\begin{tabular}{||c|l|r|l||}
	\hline
	Étiq. & Bloc & $\livein{\cdot}$ & $\liveout{\cdot}$ \\
	\hline
	1 & a := 1 & $\emptyset$ & $\{a\}$\\
	2 & b := 20 & $\{a\}$ & $\{a, b\}$\\
	3 & c := a + b & $\{a, b\}$ & $\{b\}$\\
	4 & while b < 100 do & $\{b\}$ & $\{b\}$\\
	5 & b := b + 1 & $\{b\}$ & $\{b\}$\\
	- & done & - & -\\
	6 & () & $\emptyset$ & $\emptyset$\\
	\hline
	\end{tabular}
\end{center}
De la même manière, $\liveout{3} \cap \{c\} = \emptyset$ donc cette variable est considérée comme morte dans la suite du programme.
L'algorithme transforme donc le bloc en question en $Skip$ puis calcule, de la même manière, l'analyse de vivacité.
\begin{center}
	\begin{tabular}{||c|l|r|l||}
	\hline
	Étiq. & Bloc & $\livein{\cdot}$ & $\liveout{\cdot}$ \\
	\hline
	1 & a := 1 & $\emptyset$ & $\emptyset$\\
	2 & b := 20 & $\emptyset$ & $\{b\}$\\
	3 & () & $\{b\}$ & $\{b\}$\\
	4 & while b < 100 do & $\{b\}$ & $\{b\}$\\
	5 & b := b + 1 & $\{b\}$ & $\{b\}$\\
	- & done & - & -\\
	6 & () & $\emptyset$ & $\emptyset$\\
	\hline
	\end{tabular}
\end{center}
Enfin, $\liveout{1} = \emptyset$ donc la variable $a$ est morte dans la suite du programme, le bloc d'étiquette 1 est à son tour
réduit, et l'on atteint un point fixe sur la réduction, c'est-à-dire que l'algorithme ne peut plus réduire quoi que ce soit et renvoie
toujours le même programme. On a donc le programme réduit,
\begin{lstlisting}[tabsize=2]
	b := 0;
	while b < 100 do
		b := b + 1
	done
\end{lstlisting}
\subsection{Incrémentalisation de la réduction}
Cette réduction naïve est cependant particulièrement inefficace. En effet, elle doit à chaque itération calculer
l'analyse à partir de $\bot$ et cela peut s'avérer lourd lorsque nos programmes se composent de milliers de blocs.
On introduit donc la réduction de code mort par incrémentalisation, dans laquelle l'on essaye plutôt de calculer
la nouvelle analyse de vivacité à partir de la précédente, ce qui réduit considérablement le nombre de calculs.
Pour résumer l'idée, voilà comment nous pourrions définir cette fois notre application de réduction,
\[\Delta : \mathcal{A} \times Stm \longrightarrow \mathcal{A} \times Stm \]
\noindent
\begin{notation}
	Par la suite, si $s$ est une déclaration, on notera $\reduced{s}{\mathcal{L}}{Skip}$ ce même programme, réduit aux blocs d'étiquette dans $\mathcal{L}$.
\end{notation}

Cependant, si l'on considère $s$ une déclaration, $\mu_s$ son point fixe minimal et $s' = \reduced{s}{\mathcal{L}}{Skip}$ cette même déclaration
réduite, alors il n'est pas possible d'utiliser l'analyse de vivacité de $s$ pour poursuivre la réduction, étant donné qu'on ne peut pas garantir que
$\mu_{s} \sqsubseteq \mu_{s'}$ et que, comme démontré plus haut, la recherche de point fixe est croissante monotone uniquement. 
Il s'agit donc de trouver un moyen de suffisament décroître le précédent treillis.
Pour ce faire, on peut introduire un ensemble $\filterset$ qui agit comme un filtre de manière à supprimer l'information en trop. 
De manière à filtrer convenablement le treillis de $s$, il faut ajouter de l'information à cette dernière. 
En effet, il est nécessaire qu'à chaque bloc $b$, il soit possible de connaître quel bloc sous-jacent
a besoin des variables vivantes en $b$. Pour ce faire, le treillis utilisé n'est plus adapté étant donné qu'il n'est pas en mesure
d'indiquer d'où vient la propagation d'une variable vivante. On essaye donc plutôt d'utiliser le treillis $(\mathcal{P}(\mathbb{L} \times \mathbb{V}), \sqsubseteq)$. 
\\
\\
En ce qui concerne la comparaison de ces paires, elle dépend de la manière avec laquelle l'algorithme de réduction parcourira la déclaration donnée.
En effet, si celle-ci se fait strictement de bas en haut, alors la comparaison peut uniquement se faire sur la variable. Nous préférerons néanmoins
que ce treillis serve quelque soit l'ordre dans lequel l'algorithme effectue la réduction, la comparaison ce fera donc, dans cette étude, sur l'étiquette
et sur la variable. Une fois cela acquis, il est possible d'énoncer le théorème qui suit;
\\
\\
\begin{theorem}
	Soient $s$ une déclaration, et $s' = \reduced{s}{\mathcal{L}}{Skip}$ une réduction sur
	l'ensemble d'étiquettes $\mathcal{L}$. Soient $\mu_s, \mu_{s'} \in \mathcal{A}$ les point fixes minimaux respectifs
	des deux déclarations.
	Alors $\exists \filterset$ tel que $\forall l \in \mathbb{L}$,
	\[\mu_s[l] - \filterset \sqsubseteq \mu_{s'}[l]\]
\end{theorem}
\\
\\
\begin{proof}
	Soit $s$ une déclaration. On pose $s' := s[\mathcal{L} \longmapsto Skip]$ la déclaration à partir de $s$ dont les blocs d'étiquette dans $\mathcal{L}$ sont réduits. 
	On se place en outre sur le treillis $(\mathcal{P}(\mathbb{L} \times \mathbb{V}), \sqsubseteq)$. 
	Une fois cela donné, on redéfinit ce que sont $vars$, $gen$ et $kill$ de la manière qui suit, 
	\[vars : \mathbb{L} \times Exp_a \cup Exp_b \longrightarrow \mathcal{P}(\mathbb{L} \times \mathbb{V})\]
	retourne désormais l'ensemble des variables liées chacunes à l'étiquette du bloc qui les génère.
	\begin{align*}
		gen : \mathbb{L} \times Block &\longrightarrow \mathcal{P}(\mathbb{L} \times \mathbb{V})\\
		(l, \sassign{Id}{Exp_a}) &\longmapsto vars(l, Exp_a)\\
		(l, \sskip) &\longmapsto \emptyset\\
		(l, Exp_b) &\longmapsto vars(l, Exp_b)
	\end{align*}
	génère maintenant l'ensemble $\sgen{l}$ et,
	\begin{align*}
		kill : Block &\longrightarrow \mathcal{P}(\mathbb{L} \times \mathbb{V})\\
		\sassign{Id}{Exp_a} &\longmapsto \mathbb{L} \times \{Id\}\\
		\sskip &\longmapsto \emptyset\\
		Exp_b &\longmapsto \emptyset
	\end{align*}
	génère maintenant l'ensemble $\skill{l}$.
	\\ 
	On définit par commodité les applications
	\begin{align*}
		g : \mathcal{P}(\mathbb{L} \times \mathbb{V}) &\longrightarrow \mathcal{P}(\mathbb{L} \times \mathbb{V})\\
		X_l &\longmapsto \bigcup\limits_{p\in succ(l)} \livein{p}
	\end{align*}
	monotone croissante, d'après la démonstration sur la monotonie de $\liveout{\cdot}$ ci-dessus et,
	\begin{align*}
		f : \mathcal{P}(\mathbb{L} \times \mathbb{V}) &\longrightarrow \mathcal{P}(\mathbb{L} \times \mathbb{V})\\
		X_l &\longmapsto \sgen{l} \cup (g(X_l) - \skill{l})
	\end{align*}
	également monotone croissante par la démonstration sur la monotonie de $\livein{\cdot}$ ci-dessus.
	On pose $E_s$ et $E_{s'}$ les deux systèmes d'équations respectivement de $s$ et de $s'$,
	\[
	E_s :
	\begin{dcases*}
		\livein{l} = f(\livein{l}) &$\forall l \in \mathbb{L} - \mathcal{L}$\,, \\
		\livein{k} = f(\livein{k}) &$\forall k \in \mathcal{L}$\,,
	\end{dcases*}
	\]
	et
	\[
	E_{s'} :
	\begin{dcases*}
		\livein{l} = f(\livein{l}) &$\forall l \in \mathbb{L} - \mathcal{L}$\,,\\
		\livein{k} = g(\livein{k}) &$\forall k \in \mathcal{L}$\,.
	\end{dcases*}
	\]
	Posons enfin, l'ensemble filtre tel que,
	\[
		\filterset = \{(l, v) \mid l \in \mathcal{L}, v \in \mathbb{V}\}
	\]
	On considère $S = (S_l)_{l\in \mathbb{L}}$ le plus petit point fixe de la déclaration $s$ \textit{i.e.} la plus petite solution du système $E_s$.
	Montrons d'abord que $\forall X = (X_l)_{l\in \mathbb{L}}$ solution de $E_{s'}$ alors,
	\[S_k - \filterset \sqsubseteq X_k, \forall k \in \mathcal{L}\]
	D'après la définition de $gen$ sur ce treillis il vient $\sgen{k} \sqsubseteq I_{\mathcal{L}}$, donc on a
	\[S_k - \filterset = g(S_k) - (\skill{k} - \{(k, v) \mid v \in \mathbb{V}\}) \sqsubseteq g(S_k) \sqsubseteq g(X_k) = X_k\]
	comme l'on suppose $S$ la plus petite solution du système $E_s$.
	Montrons ensuite que $\forall X = (X_l)_{l\in \mathbb{L}}$ solution de $E_{s'}$ on a,
	\[
		S_l - \filterset \sqsubseteq X_l, \forall l \in \mathbb{L} - \mathcal{L}
	\]
	Dans ce cas, on remarque que 
	\[S_l - X_l = S_l \cap \filterset\] 
	mais aussi que, 
	\[g(X_l) - g(S_l) \sqsubseteq \mathcal{U} = \bigcup\limits_{k\in \mathcal{L}} g(k) \cap \skill{k}\]
	et ainsi que,
	\[g(S_l) - I_{\mathcal{L}} \sqsubseteq g(X_l)\]
	Donc on obtient que,
	\[
		S_l - \filterset = \sgen{l} \cup (g(X_l) - \mathcal{U} - \skill{l}) \sqsubseteq f(X_l) = X_l
	\]
	Étant donné que $[g(X_l) - \mathcal{U}] \cap \filterset = \emptyset$ il n'est pas nécessaire de retirer à $\skill{l}$ les possibles
	paires d'étiquettes dans $\mathcal{L}$.
	Donc $\forall l \in \mathbb{L}$ on a bien montré que $S_l - \filterset \sqsubseteq X_l$ pour tout $X$ solution du système $E_{s'}$.
	Donc $S$ est bien un pré-point fixe de $\mu_{s'}$, la plus petite solution de $E_{s'}$.\hfill$\square$ 
\end{proof}
\subsubsection{Exemple}
Illustrons tout ce qui vient d'être dit, avec un programme sur lequel l'on fait une incrémentalisation, d'abord à l'aide du premier
treillis, puis à l'aide du second. Considérons le programme,
\begin{lstlisting}[tabsize=2]
	a := 0;
	b := a;
	while a < 100 do
		a := a + 1
		c := a + b
	done;
	d := a * b * c
\end{lstlisting}
On trouve donc la première analyse de vivacité, à partir de $\bot$ sur le treillis $(\mathcal{P}(V), \sqsubseteq)$
\begin{center}
	\begin{tabular}{||c|l|r|l||}
	\hline
	Étiq. & Bloc & $\livein{\cdot}$ & $\liveout{\cdot}$ \\
	\hline
	1 & a := 1 & $\emptyset$ & $\{a\}$\\
	2 & b := a & $\{a\}$ & $\{a, b\}$\\
	3 & while a < 100 do & $\{a, b\}$ & $\{a, b\}$\\
	4 & a := a + 1 & $\{a, b\}$ & $\{a, b\}$\\
	5 & c := a + b & $\{a, b\}$ & $\{a, b, c\}$\\
	- & done & - & -\\
	6 & d := a * b * c & $\{a, b, c\}$ & $\emptyset$\\
	\hline
	\end{tabular}
\end{center}
Supposons désormais, qu'on veuille réduire le bloc d'étiquette 5, et produire la nouvelle analyse de vivacité
à partir de celle décrite ci-dessus. En ce bloc, on a $\sgen{5} = \{a, b\}$ donc il faut propager la perte de vivacité
de ce bloc, vers ses blocs prédecesseurs. Il vient le problème de l'origine de la vivacité sur une variable. En effet,
en l'état de ce treillis, il est impossible de savoir, ni qui utilise ces variables dans les blocs successeurs ni qui
maintient la vivacité, pour cause du bloc 5, dans les blocs prédecesseurs. Essayons désormais le second treillis, 
$(\mathcal{P}(\mathbb{N} \times V), \sqsubseteq)$. Cette fois-ci, on trouve la première analyse de vivacité, à partir
de $\bot$,
\begin{center}
	\begin{tabular}{||c|l|r|l||}
	\hline
	Étiq. & Bloc & $\livein{\cdot}$ & $\liveout{\cdot}$ \\
	\hline
	1 & a := 1 & $\emptyset$ & $\{a_2, a_3, a_4\}$\\
	2 & b := a & $\{a_2, a_3, a_4\}$ & $\{a_3, a_4, b_5, b_6\}$\\
	3 & while a < 100 do & $\{a_3, a_4, b_5, b_6\}$ & $\{a_4, b_5, b_6\}$\\
	4 & a := a + 1 & $\{a_4, b_5, b_6\}$ & $\{a_3, a_4, a_5, a_6, b_5, b_6\}$\\
	5 & c := a + b & $\{a_3, a_4, a_5, a_6, b_5, b_6\}$ & $\{a_3, a_4, a_6, b_5, b_6, c_6\}$\\
	- & done & - & -\\
	6 & d := a * b * c & $\{a_6, b_6, c_6\}$ & $\emptyset$\\
	\hline
	\end{tabular}
\end{center}
Réduisons désormais le bloc d'étiquette 5 et appliquons le filtre $I_{\{5\}}[\mu_P]$ sur l'analyse,
\begin{center}
	\begin{tabular}{||c|l|r|l||}
	\hline
	Étiq. & Bloc & $\livein{\cdot}$ & $\liveout{\cdot}$ \\
	\hline
	1 & a := 1 & $\emptyset$ & $\{a_2, a_3, a_4\}$\\
	2 & b := a & $\{a_2, a_3, a_4\}$ & $\{a_3, a_4, b_6\}$\\
	3 & while a < 100 do & $\{a_3, a_4, b_6\}$ & $\{a_4, b_6\}$\\
	4 & a := a + 1 & $\{a_4, b_6\}$ & $\{a_3, a_4, a_6, b_6\}$\\
	5 & () & $\{a_3, a_4, a_6, b_6\}$ & $\{a_3, a_4, a_6, b_6, c_6\}$\\
	- & done & - & -\\
	6 & d := a * b * c & $\{a_6, b_6, c_6\}$ & $\emptyset$\\
	\hline
	\end{tabular}
\end{center}
\section{Généralisation}
\section{Annexe}
\subsection{Validation}
\subsection{Tests unitaires}
\subsection{Générateur}
Dans l'optique de vérification de la robustesse de nos différentes analyses, il peut être bon de réaliser
des tests unitaires sur chacune d'elles. Pour cela on ne se restreindra pas aux tests conçus pas nous-même
mais on essayera aussi de trouver tous les cas de figure par la manière forte. Un générateur de code complètement
arbitraire est alors une bonne solution pour réaliser de tels tests. Dans notre cas, il permettra de mettre en
lumière défaut sur l'algorithme de worklist.
\\
Ce générateur de code est ajusté par un paramètre qu'est la quantité de variables assignées du programme. On
cherche également à ce que le code généré soit le plus proche possible de celui qu'aurait pu fournir un humain.
Dans notre syntaxe, il suffit pour le moment de donner une certaine répartition lors de la création des déclarations,
en donnant plus de chance à l'assignation d'apparaître, que des blocs Ifte ou Whiledo par exemple.
\subsection{Bugs}
\section{Conclusion}
\end{document}