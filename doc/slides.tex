\documentclass{beamer}
\usepackage[francais]{babel}
\usepackage{tabularx}
\usepackage{amsmath}
\usepackage{amssymb}
\usepackage{graphicx}
\usepackage{caption}
\usepackage{listings}
\usepackage{mathtools}
\usepackage{tikz}
\usepackage{algorithm}
\usepackage{algpseudocode}
\usetheme{Warsaw}
\useinnertheme{rectangles}
\mode<presentation>

\title{Minigloca}
\author{Vladislas de Haldat}

\begin{document}
\newcommand{\deftype}[2]{
    ${#1}\rightarrow$\foreach[count=\i] \e in {#2} {
        \ifnum\i>1$\mid$\fi
        \space${\e}$\\
    }
}

\newcommand{\definition}[1] {
    \textbf{\textit{Définition --\space}}{#1}
}

\newcommand{\theorem}[1] {
    \textbf{\textit{Théorème --\space}}{#1}
}

\newcommand{\sassign}[2] {
    {#1}\space:=\space{#2}
}

\newcommand{\sseq}[2] {
    {#1}\space;\space{#2}
}

\newcommand{\sskip}{
    \textit{Skip}
}

\newcommand{\ifthenelse}[3] {
    \text{if }{#1}\text{ then }{#2}\text{ else }{#3}
}

\newcommand{\whiledo}[2] {
    \text{while }{#1}\text{ do }{#2}
}
\begin{frame}
	\titlepage
\end{frame}

\begin{frame}{Index}
	\tableofcontents
\end{frame}

\section{Définition du langage}
\subsection{Expressions arithmétiques et booléennes}
\begin{frame}
	\frametitle{Expressions arithmétique et booléenne}
	\begin{align*}
  		a & ::= n & n \in \mathbb{Z}\\
		&\mid x & x \in \mathbb{V}\\
		&\mid op_A(a_1, a_2) & op_A \in \{+, -, \times\}
	\end{align*}
	\begin{dtype}{\boolexp}
		\inlinekind{\textbf{true}}
		\kind{\textbf{false}}\\
		\akind{op_R (a_1, a_2)}\with{op_R\in\{<, =\}}
		\akind{op_B (b_1, b_2)}\with{op_B\in\{\wedge, \vee\}}
		\akind{\neg{b}}
	  \end{dtype}
\end{frame}

\subsection{Déclarations}
\begin{frame}{Déclarations}
	\begin{dtype}{Stm}
		\inlinekind{\sassign{x}{a}}\\
		\akind{\sseq{s_1}{s_2}}\\
		\akind{\sskip}\\
		\akind{\sifthenelse{b}{s_1}{s_2}}\\
		\akind{\swhiledo{b}{s}}\\
		\akind{\sreturn{a}}
	  \end{dtype}
\end{frame}

\defverbatim[colored]{\lst}{
	\begin{lstlisting}[language=caml, tabsize=2, basicstyle=\ttfamily,keywordstyle=\color{blue}]
		a := 1;
		b := 20;
		if a = 3 then
			c := 4
		else
			c := 6
		endif;
		while b < 100 do
			a := b + 1
		done;
		return c
	\end{lstlisting}
}
\begin{frame}{Exemple}
	\lst
\end{frame}

\section{Prérequis à l'analyse}
\subsection{Blocs}
\begin{frame}{Blocs}
	\begin{dtype}{Block}
		\inlinekind{\sassign{x}{a}}\\
		\akind{b}\\
		\akind{\sskip}\\
		\akind{\sreturn{a}}
	\end{dtype}
	\begin{block}{Notation}
		Si $s \in Stm$ et $l = \lambda(s)$ alors on notera $s^l \in Stm$ la déclaration munie d'une étiquette.
	\end{block}
\end{frame}

\subsection{Graphe de flot de contrôle}
\begin{frame}{Graphe de flot de contrôle}
	\begin{center}
		\input{cfg.fig}
	\end{center}
\end{frame}

\section{Analyse de vivacité}
\subsection{Description ensembliste}
\begin{frame}{Description ensembliste}
	\begin{block}{Définition}
		Soient les applications
		\[vars_a : \arithexp \longrightarrow \mathcal{P}(\mathbb{V})\]
		\[vars_b : \boolexp \longrightarrow \mathcal{P}(\mathbb{V})\]
		Les ensembles de variables présentes dans les expressions arithmétique et booléenne.
	\end{block}
\end{frame}

\begin{frame}{Description ensembliste}
	\begin{block}{Définition}
		Soit l'application
		\begin{align*}
			gen : Block &\longrightarrow \mathcal{P}(\mathbb{V})\\
			\sassign{x}{a} &\longmapsto vars_a(a)\\
			\sskip &\longmapsto \emptyset\\
			b &\longmapsto vars_b(b)\\
			\sreturn{a} &\longmapsto vars_a(a)
		\end{align*}
	\end{block}
\end{frame}

\begin{frame}{Description ensembliste}
	\begin{block}{Définition}
		Soit l'application
		\begin{align*}
			kill : Block &\longrightarrow \mathcal{P}(\mathbb{V})\\
			\sassign{x}{a} &\longmapsto \{x\}\\
			\sskip &\longmapsto \emptyset\\
			b &\longmapsto \emptyset\\
			\sreturn{a} &\longmapsto \emptyset
		\end{align*}
	\end{block}
\end{frame}

\begin{frame}{Description ensembliste}
	\begin{block}{Définition}
		Soient les ensembles
		\begin{align*}
			\livein{l} &= \sgen{l} \cup (\liveout{l} - \skill{l}),\\
			\liveout{l} &= 
			\begin{dcases*}
			\emptyset &si $succ(l) = \emptyset $\,, \\
			\bigcup\limits_{p\in succ(s)} \livein{p} &sinon\,.
			\end{dcases*}
		\end{align*}
	\end{block}
\end{frame}

\subsection{Point fixe}
\begin{frame}{Point fixe}
	\begin{block}{Théorème (Kleene)}
		Soit $(L, \sqsubseteq)$ un ordre partiellement ordonné, avec un plus petit élément  $\perp$ et soit
		une application $f : L \longrightarrow L$ monotone. Alors il existe un point fixe minimal qui est le suprémum de la suite,
		\[\perp \sqsubseteq f(\perp) \sqsubseteq f^2(\perp) \sqsubseteq \cdots \sqsubseteq f^k(\perp) \sqsubseteq \cdots\]
	\end{block}	
\end{frame}

\begin{frame}{Algorithme}
	\begin{algorithmic}
		\State {Soit $s$ la déclaration}
		\State $\mathcal{B} \leftarrow blocks(s)$
		\State {Soit $\mathbb{L}$ l'ensemble des étiquettes de $\mathcal{B}$}
		\For{$l \in \mathbb{L}$}
		\State $live_{in}[l] \leftarrow \emptyset$
		\State $live_{out}[l] \leftarrow \emptyset$
		\EndFor
		\While{$live_{in} \ne live_{in}' \textbf{ ou } live_{out} \ne live_{out}'$}
		\For{$l \in L$}
		\State $live_{out}[l] \leftarrow \bigcup\limits_{p\in succ(l)} live_{in}[p]$
		\State $live_{in}[l] \leftarrow \sgen{l} \cup (live_{out}[l] - \skill{l})$
		\EndFor
		\EndWhile
	\end{algorithmic}
\end{frame}

\defverbatim[colored]{\lst}{
	\begin{lstlisting}[language=caml, tabsize=2, basicstyle=\ttfamily,keywordstyle=\color{blue}]
		a := 0;						// empty		{a}
		b := a;						// {a}			{a, b}
		while a < 100 do	// {a, b}		{a, b}
			if a = 2 then		// {a, b}		{a, b}
				c := a				// {a, b}		{b, c}
			else
				c := 2 * a;		// {a, b}		{b, c}
				d := b				// {b, c}		{b, c}
			endif;
			a := c + 1			// {b, c}		{a, b, c}
		done;						
		return c					// {c}			empty
	\end{lstlisting}
}
\begin{frame}{Exemple}
	\lst
\end{frame}

\section{Élimination de code mort}
\subsection{Réduction naïve}
\begin{frame}{Réduction naïve}
	\begin{block}{Définition}
		Soit $(x := a)^l$ un bloc d'affectation où $x \in \mathbb{V}$ et $l \in \mathbb{L}$. Si $x \notin \liveout{l}$, on dit que la variable $x$ est morte.
	\end{block}
	\begin{block}{Approche}
		Notons $\mathcal{A} = \mathcal{P}(\mathbb{V})^2$ et soit l'application
		\[\Delta : \mathcal{A} \times Stm \longrightarrow Stm\]
		On recalcule l'analyse à partir de $\bot$.
	\end{block}
\end{frame}

\begin{frame}{Exemple}
	\lst
\end{frame}

\defverbatim[colored]{\lst}{
	\begin{lstlisting}[language=caml, tabsize=2, basicstyle=\ttfamily,keywordstyle=\color{blue}]
		a := 0;						// empty	{a}
		b := a;						// {a}		{a}
		while a < 100 do	// {a}		{a}
			if a = 2 then		// {a}		{a}
				c := a				// {a}		{c}
			else
				c := 2 * a;		// {a}		{c}
				skip					// {c}		{c}
			endif;
			a := c + 1			// {c}		{a, c}
		done;						
		return c					// {c}		empty
	\end{lstlisting}
}
\begin{frame}{Exemple}
	\lst
\end{frame}

\defverbatim[colored]{\lst}{
	\begin{lstlisting}[language=caml, tabsize=2, basicstyle=\ttfamily,keywordstyle=\color{blue}]
		a := 0;						// empty	{a}
		skip;							// {a}		{a}
		while a < 100 do	// {a}		{a}
			if a = 2 then		// {a}		{a}
				c := a				// {a}		{c}
			else
				c := 2 * a;		// {a}		{c}
				skip					// {c}		{c}
			endif;
			a := c + 1			// {c}		{a, c}
		done;						
		return c					// {c}		empty
	\end{lstlisting}
}
\begin{frame}{Exemple}
	\lst
\end{frame}

\subsection{Incrémentalisation}
\begin{frame}{Incrémentalisation}
	\begin{itemize}
		\item Mal optimisé : calcul d'un point fixe à partir de $\bot$ à chaque itération
		\item Est-il possible de réutiliser la précédente analyse ?
		\item La monotonie sera-t-elle conservée ?
	\end{itemize}
\end{frame}

\defverbatim[colored]{\lst}{
	\begin{lstlisting}[language=caml, tabsize=2, basicstyle=\ttfamily,keywordstyle=\color{blue}]
		a := 0;						// empty		{a}
		b := a + 1;				// {a}			{a, b}
		c := 2 * b;				// {a, b}		{a}
		return a					// {a}			empty
	\end{lstlisting}
}
\begin{frame}{Exemple}
	\lst
\end{frame}

\defverbatim[colored]{\lst}{
	\begin{lstlisting}[language=caml, tabsize=2, basicstyle=\ttfamily,keywordstyle=\color{blue}]
		a := 0;						// empty	{a}
		b := a + 1;				// {a}		{a}
		skip;							// {a}		{a}
		return a					// {a}		empty
	\end{lstlisting}
}
\begin{frame}{Exemple}
	\lst
	\begin{block}{Remarque}
		Cette analyse de vivacité est plus petite que la précédente, on perd la monotonie!
	\end{block}
\end{frame}

\begin{frame}{Incrémentalisation}
	\begin{itemize}
		\item Perte de la monotonie en prenant l'analyse précédente.
		\item Il faut alors pouvoir la réduire suffisament.
		\item Est-il possible d'y appliquer un filtre ?
	\end{itemize}
\end{frame}

\defverbatim[colored]{\lst}{
	\begin{lstlisting}[language=caml, tabsize=2, basicstyle=\small,keywordstyle=\color{blue}]
		a := 0;						// empty		{a}
		b := a;						// {a}			{a, b}
		b := b + 3;				// {a, b}		{a, b}
		while a < 100 do	// {a, b}		{a, b}
			if a = 2 then		// {a, b}		{a, b}
				c := a;				// {a, b}		{b, c}
				d := b				// {b, c}		{b, c}
			else
				c := 2 * a;		// {a, b}		{b, c}
				e := b				// {b, c}		{b, c}
			endif;
			a := c + 1			// {b, c}		{a, b, c}
		done;
		return c					// {c}			empty
	\end{lstlisting}
}
\begin{frame}{Exemple}
	\lst
\end{frame}

\defverbatim[colored]{\lst}{
	\begin{lstlisting}[language=caml, tabsize=2, basicstyle=\small,keywordstyle=\color{blue}, escapeinside={(*}{*)}]
		a := 0;						// empty	{(*$a_4,a_5,a_6,a_8$*)}
		b := a;						// {(*$a_4,a_5,a_6,a_8$*)}	{(*$a_4,a_5,a_6,a_8,b_3$*)}
		b := b + 3;				// {(*$a_4,a_5,a_6,a_8,b_3$*)}	{(*$a_4,a_5,a_6,a_8,b_7,b_9$*)}
		while a < 100 do	// {(*$a_4,a_5,a_6,a_8,b_7,b_9$*)}	{(*$a_5,a_6,a_8,b_7,b_9$*)}
			if a = 2 then		// {(*$a_5,a_6,a_8,b_7,b_9$*)}	{(*$a_6,a_8,b_7,b_9$*)}
				c := a;				// {(*$a_6,b_7,b_9$*)}	{(*$b_7,b_9,c_{10},c_{11}$*)}
				d := b				// {(*$b_7,b_9,c_{10},c_{11}$*)}	{(*$b_7,b_9,c_{10},c_{11}$*)}
			else
				c := 2 * a;		// {(*$a_8,b_7,b_9$*)}	{(*$b_7,b_9,c_{10},c_{11}$*)}
				e := b				// {(*$b_7,b_9,c_{10},c_{11}$*)}	{(*$b_7,b_9,c_{10},c_{11}$*)}
			endif;
			a := c + 1			// {(*$b_7,b_9,c_{10},c_{11}$*)}	{(*$a_4,a_5,a_6,a_8,b_7,b_9,c_{11}$*)}
		done;
		return c					// {(*$c_{11}$*)}	empty
	\end{lstlisting}
}
\begin{frame}{Exemple}
	\lst
\end{frame}

\begin{frame}{Incrémentalisation}
	On travaille désormais sur le treillis $(\mathcal{P}(\mathbb{L}\times\mathbb{V}), \subseteq)$.
	\begin{block}{Notation}
		\[
			\mathcal{L} = \{l \in \mathbb{L} \mid (x := a)^l \text{ et } \{x\} \notin \liveout{l}\}	
		\]
		Si $s \in Stm$ une déclaration, on notera $\reduced{s}{\mathcal{L}}{\sskip}$ cette même déclaration, 
		réduite aux blocs d'étiquette dans $\mathcal{L}$.
	\end{block}
\end{frame}

\begin{frame}{Incrémentalisation}
	On redéfinit l'analyse sur ce treillis.
	\begin{align*}
		gen : \mathbb{L} \times Block &\longrightarrow \mathcal{P}(\mathbb{L} \times \mathbb{V})\\
		(l, \sassign{x}{a}) &\longmapsto vars_a(l, a)\\
		(l, \sskip) &\longmapsto \emptyset\\
		(l, b) &\longmapsto vars_b(l, b)\\
		(l, \sreturn{a}) &\longmapsto vars_a(l, a)
	\end{align*}
\end{frame}

\begin{frame}{Incrémentalisation}
	\begin{align*}
		kill : Block &\longrightarrow \mathcal{P}(\mathbb{L} \times \mathbb{V})\\
		\sassign{x}{a} &\longmapsto \mathbb{L} \times \{x\}\\
		\sskip &\longmapsto \emptyset\\
		b &\longmapsto \emptyset\\
		\sreturn{a} &\longmapsto \emptyset
	\end{align*}
\end{frame}

\begin{frame}{Incrémentalisation}
	\begin{block}{Théorème}
		Soient $s \in Stm$, et $s' = \reduced{s}{\mathcal{L}}{\sskip}$ une réduction sur
	l'ensemble d'étiquettes $\mathcal{L}$. Soient $\mu_s$ et $\mu_{s'}$ les point fixes minimaux respectifs
	des deux déclarations.
	Alors il existe $\filterset$ tel que $\forall l \in \mathbb{L}$,
	\[
		\mu_s[l] - \filterset \subseteq \mu_{s'}[l]
	\]
	est un pré-point fixe de $\mu_{s'}[l]$.
	\end{block}
	\begin{block}{Remarque}
		On prendra $\filterset = \mathcal{L}\times\mathbb{V}$
	\end{block}
\end{frame}

\defverbatim[colored]{\lst}{
	\begin{lstlisting}[language=caml, tabsize=2, basicstyle=\ttfamily,keywordstyle=\color{blue},escapeinside={(*}{*)}]
		a := 0;						// empty		{(*$a_2,a_4$*)}
		b := a + 1;				// {(*$a_2,a_4$*)}		{(*$a_4,b_3$*)}
		c := 2 * b;				// {(*$a_4, b_3$*)}		{(*$a_4$*)}
		return a					// {(*$a_4$*)}			empty
	\end{lstlisting}
}
\begin{frame}{Exemple}
	Reprenons le premier exemple:
	\lst
	\begin{block}{Remarque}
		$\forall l \in \mathbb{L}$ on a $(l, c) \notin \{(4, a)\}$ donc $c$ est morte.
	\end{block}
\end{frame}

\defverbatim[colored]{\lst}{
	\begin{lstlisting}[language=caml, tabsize=2, basicstyle=\ttfamily,keywordstyle=\color{blue},escapeinside={(*}{*)}]
		a := 0;						// empty		{(*$a_2,a_4$*)}
		b := a + 1;				// {(*$a_2,a_4$*)}		{(*$a_4$*)}
		skip;							// {(*$a_4$*)}			{(*$a_4$*)}
		return a					// {(*$a_4$*)}			empty
	\end{lstlisting}
}
\begin{frame}{Exemple}
	\lst
\end{frame}

\defverbatim[colored]{\lst}{
	\begin{lstlisting}[language=caml, tabsize=2, basicstyle=\ttfamily,keywordstyle=\color{blue},escapeinside={(*}{*)}]
		a := 0;						// empty		{(*$a_4$*)}
		skip;							// {(*$a_4$*)}			{(*$a_4$*)}
		skip;							// {(*$a_4$*)}			{(*$a_4$*)}
		return a					// {(*$a_4$*)}			empty
	\end{lstlisting}
}
\begin{frame}{Exemple}
	\lst
\end{frame}

\defverbatim[colored]{\lst}{
	\begin{lstlisting}[language=caml, tabsize=2, basicstyle=\ttfamily,keywordstyle=\color{blue},escapeinside={(*}{*)}]
		a := 0;		
		#if (*$\{b\}\in\livein{3}$*)			
			b := a + 1;
		#if (*$\{c\}\in\livein{4}$*)	
			c := 2 * b;	
		return a
	\end{lstlisting}
}
\section{Généralisation}
\begin{frame}{Généralisation}
	Avec un prédicat sur la vivacité des variables.
	\lst
\end{frame}

\defverbatim[colored]{\lst}{
	\begin{lstlisting}[language=c, tabsize=2, basicstyle=\ttfamily,keywordstyle=\color{blue},escapeinside={(*}{*)}]
		int div(int x)
		{
			return x/32;
		}
	\end{lstlisting}
}
\defverbatim[colored]{\lsta}{
	\begin{lstlisting}[tabsize=2, basicstyle=\ttfamily,keywordstyle=\color{blue},escapeinside={(*}{*)}]
		div:
			mov     eax, DWORD PTR [rbp-4]
			lea     edx, [rax+31]
			test    eax, eax
			cmovs   eax, edx
			sar     eax, 5
			ret
	\end{lstlisting}
}
\begin{frame}{Généralisation}
	Considérons le programme
	\lst
	Son code assembleur, produit par GCC
	\lsta
\end{frame}

\defverbatim[colored]{\lsta}{
	\begin{lstlisting}[tabsize=2, basicstyle=\ttfamily,keywordstyle=\color{blue},escapeinside={(*}{*)}]
		div:
			mov     eax, DWORD PTR [rbp-4]
			shr     eax, 5
			ret
	\end{lstlisting}
}
\begin{frame}{Généralisation}
	En supposant l'entier toujours positif
	\lsta
	Posons alors le prédicat
	\[P : v \longmapsto \intrfun{v > 0 \vee v = 0}{B}(\sigma)\]
\end{frame}

\end{document}
