\subsection{Blocs}
De manière à faciliter l'analyse d'un programme, il est utile de partitionner et de factoriser notre représentation du code. 
Les blocs ont été évoqués dans la partie précédente comme des sous-parties de l'ensemble des déclarations atomiques d'un programme. 
On peut désormais les définir de la sorte,

\begin{dtype}{Block}
	\inlinekind{\sassign{Id}{Exp_a}}\\
	\akind{Exp_b}\\
	\akind{\sskip}
\end{dtype}
Maintenant les blocs correctement définis, il nous faut pouvoir les lier à une étiquette. Pour ce faire, on se donne $\beta$ définie par,

\begin{align*}
	\beta : Stm &\longrightarrow \mathcal{P}(\mathbb{N} \times Block)\\
	(\sassign{Id}{Exp_a})^l&\longmapsto\{(l, Id := Exp_a)\}\\
	\sseq{s_1}{s_2} &\longmapsto \beta(s_1) \cup \beta(s_2)\\
	\sskip^l &\longmapsto \{(l, \textit{Skip})\}\\
	\sifthenelse{b^l}{s_1}{s_2} &\longmapsto \{(l, b)\}\cup \beta(s_1)\cup\beta(s_2)\\
	\swhiledo{b^l}{s} &\longmapsto \{(l, b)\}\cup\beta(s)
\end{align*}
À partir de là, il est possible de formaliser les graphes orientés de flot. Les blocs en représentent les noeuds, et le passage 
vers le bloc suivant est représenté par un arc.